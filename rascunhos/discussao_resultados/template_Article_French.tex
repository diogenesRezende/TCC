\documentclass[french]{article}
\usepackage[T1]{fontenc}
\usepackage[utf8]{inputenc}
\usepackage{lmodern}
\usepackage[a4paper]{geometry}
\usepackage{babel}
\begin{document}
\section{Discussão de Resultados}

	\par Neste capítulo serão discutidos os principais pontos em relação 
aos resultados obtidos com a execução desta pesquisa. Espera-se com isso
elucidar algumas questões referentes ao modo como a aplicação das teorias
descritas no quadro teórico desta, refletiram na prática.\\\\
	\par \textbf{parte diego}\\\\

	\par O aplicativo resultado desta pesquisa, tinha necessidade de
consumir dados para posteriormente apresentá-los ao usuário. Era 
necessário que, os dados do sistema acadêmico da instituição de ensino 
que serviu como contexto para esta pesquisa, fossem transmitidos de alguma
forma ao aplicativo. Era necessário também que os dados chegassem ao 
aplicativo respeitando as particularidades de cada usuário, trazendo somente 
informações relevantes aos mesmos. Com esse intuito de disponibilizar 
informações já citadas anteriormente, a quem quer que fosse necessário, 
inclusive aos usuários do aplicativo, foi criado um \textit{Web Service}.
	\par  

	
	%facilidade de comunicação com web service
	%rest
	%Módulo de comunicação
	


\end{document}
