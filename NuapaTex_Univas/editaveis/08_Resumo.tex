% --- resumo em português ---

\begin{OnehalfSpacing} 

\noindent \imprimirAutorCitacaoMaiuscula. {\bfseries\imprimirtitulo}.
 {\imprimirdata}.  Monografia -- Curso de {\MakeUppercase\imprimircurso},
 {\imprimirinstituicao}, {\imprimirlocal}, {\imprimirdata}.

\vspace{\onelineskip}
\vspace{\onelineskip}
\vspace{\onelineskip}
\vspace{\onelineskip}

\begin{resumo}
~\\
%início do texto do resumo
\noindent Atualmente, existe uma grande quantidade de tecnologias disponíveis
no mercado e sabe-se também que, com a popularização dos \textit{smartphones} e
\textit{tablet} tornou-se comum utilizar dispositivos móveis para receber
informações em tempo real. Neste contexto, nesta pesquisa foi desenvolvido um
\textit{web service} para disponibilizar as informações aos discentes da Univás
através de serviços. Também foi criado um aplicativo na plataforma Android que
consome o serviço do \textit{web service}. O \textit{app} permite aos alunos da
Universidade do Vale do Sapucaí consultarem suas informações acadêmicas como
notas, faltas e provas agendadas do semestre corrente. No momento em que um
professor lançar um evento, o servidor transmite a mensagem para a API Google
Cloud Messaging (GCM) que se responsabiliza em entregar os dados para o
aplicativo Android. Este trabalho enquadra-se no tipo de pesquisa aplicada,
pois foi desenvolvido um produto real com propósito de resolver um problema
especifico. A plataforma Android foi escolhida pelo seu destaque no mercado.

%fim do texto do resumo
\vspace{\onelineskip}
\vspace*{\fill}
\noindent \textbf{Palavras-chave}: \imprimirPalavraChaveUm. \imprimirPalavraChaveDois. \imprimirPalavraChaveTres.
\vspace{\onelineskip}
\end{resumo}

\end{OnehalfSpacing}
