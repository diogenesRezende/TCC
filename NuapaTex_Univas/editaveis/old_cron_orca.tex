\section{Participantes}

	\par Os participantes serão responsáveis por planejar, executar e testar o
\textit{software} a ser desenvolvido. A seguir serão descritos os integrantes
que desenvolverão esta pesquisa.

	\par Diego D’leon Nunes, técnico em informática formado pelo INPETTECC, é aluno
do VII período do curso de sistemas de informação da universidade do vale do
sapucaí. Atualmente desempenha a função de analista de suporte na empresa
Automação e cia.

	\par Diógenes Aparecido Rezende, é aluno do VII período do curso de sistemas de
informação da universidade do vale do sapucaí. Atualmente desempenha a função de 
analista de suporte técnico na empresa NGTec Soluções em tecnologia LTDA.

	\par Henrique Almeida Versiani Murta, é aluno do VII período do curso de
sistemas de informação da universidade do vale do sapucaí. Atualmente
desempenha a função de assistente administrativo em compras na empresa A
Construtora Pouso Alegre LTDA.

	\par Professor Roberto Ribeiro Rocha, graduado em Ciência da Computação pela
faculdade de Administração e Informática – FAI (2002), possui especialização em
Produção de Software Livre pela Universidade Federal de Lavras – UFLA (2006) e
mestrado em Ciência e Tecnologia da Informação na Universidade Federal de
Itajubá – UNIFEI (2013). Foi analista de sistemas na Liveware Tecnologia a
Serviço Ltda e integrante da equipe de TI na Megatron Fios e Cabos Especiais.
Possui experiência na área de ciência da computação, com ênfase em arquitetura
de sistemas de computação, \textit{software} livre e Linux. Atualmente é
professor no curso de Sistemas de Informação na Univás.
\pagebreak
\section{Cronograma}

	\par Um cronograma remete-se a mostrar as etapas que se sucederão na
elaboração de um projeto. Segundo \citeonline{prodanov2013} no cronograma, você
dimensiona cada uma das etapas do desenvolvimento da pesquisa, no tempo
disponível para sua execução. Geralmente os cronogramas são divididos em meses.
Nessa seção pretende-se então, mostrar a divisão dos passos que seram
percorridos em relação ao tempo disponível, para a devida e completa execução
desta pesquisa.

\begin{table} [h]
  \caption[Cronograma de Atividades]
          {Cronograma de Atividades}
  \centering
  \begin{small}
  \setlength{\tabcolsep}{1pt} 
  \begin{tabular}{|p{3.2in}|c|c|c|c|c|c|c|c|c|c|c|c|}
    \hline
     
    \multicolumn{1}{|c|}{\textbf{Ações}}&
    \multicolumn{1}{ c|}{\textbf{Jan}}  & 
    \multicolumn{1}{ c|}{\textbf{Fev}}  &
    \multicolumn{1}{ c|}{\textbf{Mar}}  &
    \multicolumn{1}{ c|}{\textbf{Abr}}  &
    \multicolumn{1}{ c|}{\textbf{Mai}}  &
    \multicolumn{1}{ c|}{\textbf{Jun}}  &
    \multicolumn{1}{ c|}{\textbf{Jul}}  &
    \multicolumn{1}{ c|}{\textbf{Ago}}  &
    \multicolumn{1}{ c|}{\textbf{Set}}  &
    \multicolumn{1}{ c|}{\textbf{Out}}  &
    \multicolumn{1}{ c|}{\textbf{Nov}}  &
    \multicolumn{1}{ c|}{\textbf{Dez}}  \\\hline\hline 
    
    Pesquisa e levantamento de dados      &X&X&&&&&&&&&&\\\hline
    
    Escrita pré-projeto                   &X&X&&&&&&&&&&\\\hline
    
    Levantamento de requisitos funcionais do \textit{software}
    &&X&&&&&&&&&&\\\hline
    
    Escrita do quadro teórico   &&X&X&&&&&&&&&\\\hline
    
    Entrega do quadro teórico   &&&X&&&&&&&&&\\\hline
    
    Escrita do quadro metodológico      &&&X&X&&&&&&&&\\\hline
    
    Entrega do quadro metodológico      &&&&X&&&&&&&&\\\hline 
    
    Apresentação do projeto para a banca de qualificação &&&&&X&&&&&&&\\\hline
    
    Correções textuais para o TCC &&&&&X&X&&&&&&\\\hline
    
    Início do desenvolvimento do \textit{software} &&&&&&X&&&&&&\\\hline
    
    Escrita do TCC &&&&&&X&X&X&X&&&\\\hline
    
    Desenvolvimento do \textit{software} proposto &&&&&&X&X&X&X&&&\\\hline
    
    Apresentação do projeto para pré-banca &&&&&&&&&&X&&\\\hline
    
    Correções textuais do TCC &&&&&&&&&&X&X&\\\hline
    
    Apresentação do TCC para banca e para o público &&&&&&&&&&&X&\\\hline
    
    Correções textuais finais do TCC &&&&&&&&&&&X&X\\\hline
    
    Entrega do TCC &&&&&&&&&&&&X\\\hline
  
  \end{tabular}
  
  \end{small}
  
\end{table}

	\par A partir desse cronograma podemos organizar melhor as etapas desse
trabalho, fazendo uma estimativa de entrega todos os itens propostos durante o
tempo total do projeto. Conseguindo maximizar os itens relacionados no
cronograma, desde a pesquisa e levantamentos de dados até a ultima entrega.



\pagebreak
\section{Orçamento}

	\par O orçamento, onde quer que seja utilizado, é um demonstrativo do que será
gasto ao decorrer do projeto ou pesquisa. Ao fazer uma amostragem de custo,
ajuda a apresentar uma estimativa do que será gasto em relação, à total execução
do projeto. O orçamento é de suma importancia no fator economia, haja vista que
ajuda a discernir os custos pontuais do projeto.

	\par Segundo \citeonline{prodanov2013} O orçamento distribui os gastos
previstos com a pesquisa tanto em relação ao pessoal quanto com material
(também contempla a fase da elaboração do projeto, a execução da pesquisa e a
elaboração do trabalho de conclusão).

\begin{table} [h]
  \caption[Orçamento]
          {Orçamento}
  \centering
  \begin{small}
  \setlength{\tabcolsep}{1pt}
  \begin{tabular}{|p{3.5in}|c|c|c|}
    \hline 
    \multicolumn{1}{|c|}{\textbf{Material}} & 
    \multicolumn{1}{c|}{\textbf{Quantidade}} &
    \multicolumn{1}{c|}{\textbf{Valor Unitário}} &
    \multicolumn{1}{c|}{\textbf{Total}}\\\hline\hline 
    
    Livro Google Android-Casa do Código &1 &R\$ 69,90 &R\$ 69,90 \\\hline
    
    Livro REST-Casa do Código &1 &R\$ 69,90 &R\$ 69,90 \\\hline
    
    Livro Java Web Services: Up and Running   &1   &R\$ 128,50 &R\$ 128,50 \\\hline
    
    Livro Web Services With Java   &1   &R\$ 103,50 &R\$ 103,50 \\\hline
    
    Livro SOA aplicado - Casa do Codigo  &1   &R\$ 69,90 &R\$ 69,90 \\\hline 
    
    Encadernações &10   &R\$ 1,50 &R\$ 15,00 \\\hline
    
    Impressões            &1200  &R\$ 0,25 &R\$ 300,00 \\\hline
    
    Xerox &1500  &R\$ 0,15 &R\$ 225,00 \\\hline
    
    Tranportes referentes à pesquisa &1  &R\$ 200,00 &R\$ 200,00 \\\hline
    
    Serviços de Terceiros &1  &R\$ 400,00 &R\$ 400,00 \\\hline
    
    Outros &1  &R\$ 200,00 &R\$ 200,00 \\\hline
    
    \textbf{Total}        &-   &-         &\textbf{R\$ 1781,70} \\\hline
    
  \end{tabular}
  \end{small}
  
\end{table}

	\par Após os devidos levantamentos relativos aos custos estimados do projeto,
pode-se angariar recursos financeiros para a devida execução do mesmo. Pode-se
também obter um melhor manejo dos recursos disponíveis. Esse tipo de método
utilizado, mostra-se vital para qualquer tipo de projeto, desde uma pesquisa
acadêmica, até mesmo um projeto de grande porte onde se envolvam valores reais
muito altos. Deste modo a tabela apresentada acima representa uma estimativa de custos
deste projeto.