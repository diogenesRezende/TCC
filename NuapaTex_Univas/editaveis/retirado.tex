\par Para que houvesse dados para que o \textit{web service} transmitisse para
o aplicativo, era necessário receber os dados do sistema acadêmico da referida
instituição, haja vista que o \textit{web service} é independente do sistema
acadêmico desta instituição. Para esse propósito foi construído um módulo que
era responsável por fazer a importação dos dados necessários para a base de
dados do \textit{web service}. Este por sua vez tinha a responsabilidade de
fazer a importação dos dados periodicamente, e ainda tratar os tipos de dados
recebidos para tipos aplicáveis ao banco de dados local. Além disso era
preciso notificar o módulo responsável por invocar o serviço \textit{Google
Cloud Messaging} para que os dispositivos dos alunos aos quais houveram
atualizações nos dados, fossem notificados e fizessem acesso ao \textit{web
service} para solicitar esses dados atualizados. Pode-se dizer então, que este
módulo desempenhou perfeitamente seu papel, pois, com ele era facil manter os
discentes atualizados em relação aos seus dados e ao banco de dados do
\textit{web service} também atualizado e com uma estrutura que fosse condizente
com seus propósitos preestabelecido.