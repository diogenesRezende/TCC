
\chapter{DISCUSSÃO DE RESULTADOS} 

	\par Neste capítulo serão discutidos os principais pontos em relação 
aos resultados obtidos com a execução desta pesquisa. Espera-se com isso
elucidar algumas questões referentes ao modo como a aplicação das teorias
descritas no quadro teórico desta, refletiram na prática.\\\\
	\par \textbf{parte diego}\\\\

	\par O aplicativo resultado desta pesquisa, tinha necessidade de
consumir dados para posteriormente apresentá-los ao usuário. Era 
necessário que, os dados do sistema acadêmico da instituição de ensino 
que serviu como contexto para esta pesquisa, fossem transmitidos de alguma
forma ao aplicativo. Era necessário também que os dados chegassem ao 
aplicativo respeitando as particularidades de cada usuário, trazendo somente 
informações relevantes aos mesmos. Com esse intuito de disponibilizar 
informações já citadas anteriormente, a quem quer que fosse necessário, 
inclusive aos usuários do aplicativo, foi criado um \textit{Web Service} REST.
Este foi um dos resultados alcançados através desta pesquisa.
	
	\par A construção do \textit{web service}, de ínicio, mostrava-se um tanto
quanto custosa, devido a restrições das tecnologias que foram escolhidas. Por se
tratar de uma simples \textit{web service} que seria disponibilizado para suplir
a demanda de dados do aplicativo, os primeiros serviços foram construídos e
disponibilizados fazendo uso de \textit{servlets} simples e conexão
JDBC\footnote{JDBC - \textit{Java Database Connectivity}}. Este modo como foi
pensado inicialmente, era simples de ser contruído e de uma
\textit{performance} aceitável. Porém de acordo com o crescimento da demanda do
serviço, tornou-se inviável a contrução do mesmo com estas técnologias, devido
a complexidade com que era necessário contruir os serviços, haja vista que, com
estas técnologias era necessário que se fosse configurado praticamente tudo de
forma manual inclusive tratamento de erros da aplicação, respostas as
requisições e tipos de dados .Esta etapa teve, portanto, um resultado aceitavel
do ponto de vista da construção do \textit{web service}. No entanto se for
analizado do ponto de vista do conhecimento, teve um resulatdo satisfatório,
pois, foi buscando melhores formas de ??

	
	%facilidade de comunicação com web service
	%rest
	%Módulo de comunicação


