\section{Instrumentos}

	\par Pode-se dizer que um questionário é uma forma de coletar informações
através de algumas perguntas feitas a um público específico. Segundo
\citeonline{gunther2003}, o questionário pode ser definido como um conjunto
de perguntas que mede a opinião e interesse do respondente.

	\par Foi realizado um questionário simples, que está apresentado na Figura
\ref{fig:qm1}, contendo quatro perguntas e enviado para \textit{e-mails} de
alguns alunos da universidade. O foco desse questionário era saber o motivo pelo qual
os usuários mais acessavam o portal do aluno e se tinham alguma dificuldade em
encontrar o que procuravam. Obteve-se um total de treze respostas, no qual
pode-se perceber que a maioria dos entrevistados afirmam terem dificuldades
para encontrar o que precisam e que o sistema não avisa quando ocorre alguma
alteração. Sobre o motivo do acesso cem por cento respondeu que entram no
sistema web para consultar suas notas.
\pagebreak
\begin{figure}[h!]
	\centerline{\includegraphics[scale=0.5]{./imagens/2_q_metodologico/qm1.png}}
	\caption[Quetionário Aplicado]{Quetionário Aplicado.
		\textbf{Fonte:}Elaborado pelos autores.}
	\label{fig:qm1}
\end{figure}
	

	\par Outro instrumento utilizado para realizar esta pesquisa foram as
reuniões, ou seja, reunir-se com uma ou mais pessoas em um local, físico ou
remotamente para tratar algum assunto específico. Para
\citeonline{ferreira1999}, reunião é o ato de encontro entre algumas pessoas em
um determinado local, com finalidade de tratar qualquer assunto.

	\par Durante a pesquisa, foram realizadas reuniões entre os participantes com
o objetivo de discutir o andamento das tarefas pela qual cada integrante ficou
responsável. Além disso entravam em discussão, nessas reuniões, o cumprimento
das metas propostas por cada participante e o estabelecimento de novas metas.
Foram utilizadas nesta pesquisa, referências de livros, revistas, manuais e
\textit{web sites}.