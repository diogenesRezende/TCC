%\subsection{Modelagem}
	\par Ao decidir-se por esta pesquisa foi preciso entender os pontos
necessários para o bom funcionamento do software. As soluções geradas por este
trabalho tendem a atender os seguintes requisitos:

	\begin{itemize}
		
		\item Disponibilidade: O \textit{web service} foi projetado para criar uma
		estrutura pela qual a universidade pudesse disponibilizar informações através
		de serviços. Por isso, o servidor não pode ficar impossibilitado de responder
		as requisições por causa de falhas no sistema desenvolvido.
		
		\item Simplicidade: Segundo as respostas obtidas através do questionário
		aplicado, os estudantes encontram dificuldades para encontrar o que procuram
		quando acessam o portal do aluno. Portanto, o aplicativo Android, deve possuir
		uma interface simples e objetiva para levar as informações aos usuários.
		
		\item Notificação: Geralmente, o aluno acessa várias vezes ao dia o portal do
		aluno para saber se foi postada sua nota. Pensando nisso, o software precisa
		avisá-lo através de uma notificação quando uma nova informação for lançada no
		portal do aluno e ao clicar nesta notificação, devem ser apresentados a ele os
		dados lançados.
		
		\item Velocidade: Os dispositivos móveis conseguem informações em tempo real.
		Também, estando ciente de que as pessoas utilizam os dispositivos móveis para
		ganhar tempo, é indispensável que as informações cheguem ao aluno o mais breve
		possível.
	
	\end{itemize}

	\par Tendo estes paradigmas em mente, passou-se a desenvolver o software, como
pode ser acompanhado nas seções seguintes.
