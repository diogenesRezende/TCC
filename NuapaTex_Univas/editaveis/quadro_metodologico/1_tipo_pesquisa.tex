%\section{Tipo de pesquisa}
	
	\par Uma pesquisa é o ato de buscar e procurar pela resposta de algo.
\citeonline[p. 15]{markoni2002} definem pesquisa como “uma indagação minuciosa
ou exame crítico e exaustivo na procura de fatos e princípios”.

	\par Existem diversos tipos de pesquisa, no entanto percebeu-se que para o
propósito desta, a mais indicada foi a pesquisa aplicada, pois está se
desenvolvendo um projeto real que poderá ser utilizado por qualquer instituição
de ensino, mas que não mudará a forma com que as pessoas recebam suas
informações, apenas acrescentará mais uma opção de consultá-las.

	\par Segundo \citeonline[p. 15]{markoni2002}, uma pesquisa do tipo aplicada
“caracteriza-se por seu interesse prático, isto é, que os resultados sejam
aplicados ou utilizados, imediatamente, na solução de problemas que ocorrem na
realidade”.

	\par Dessa maneira, percebeu-se que esta pesquisa enquadra-se no tipo de pesquisa
aplicada, pois com a execução da mesma resolve um problema específico, e para
isso está desenvolvendo-se um aplicativo para dispositivos móveis que facilitará aos
graduandos acessarem o sistema \textit{web} de uma universidade.