%\section{Tipo de pesquisa}
	
	\par \citeonline[p.15]{markoni2002} definem pesquisa como “uma indagação minuciosa
ou exame crítico e exaustivo na procura de fatos e princípios”.
\citeonline{goncalves2008}, por sua vez, conclui que uma pesquisa constitui-se
em um conjunto de procedimentos visando alcançar o conhecimento de algo.

	\par Segundo \citeonline[p.15]{markoni2002}, a pesquisa do tipo aplicada
"caracteriza-se por seu interesse prático, isto é, que os resultados sejam
aplicados ou utilizados, imediatamente, na solução de problemas que ocorrem na
realidade".

	\par Dessa maneira, este trabalho enquadra-se no tipo de pesquisa aplicada,
pois foi desenvolvido um produto real com intuito de resolver um problema
específico, que no caso foi um \textit{web service} para que a Univás possa
disponibilizar seus dados através de serviços e um aplicativo para plataforma
Android que permita aos alunos da Universidade do Vale do Sapucaí, consultarem
suas notas, faltas e provas agendadas.
