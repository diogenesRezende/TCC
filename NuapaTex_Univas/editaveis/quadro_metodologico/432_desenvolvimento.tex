%\subsubsection{Desenvolvimento}

	%01 - Criação do database
	
	\par Com o ambiente de desenvolvimento pronto, começou de fato o
desenvolvimento. Primeiramente foi necessário criar o banco de dados no SGDB.
Este por sua vez foi criado com a ajuda do PgAdmin que é um software gráfico
para administração do SGDB, e que fornece uma interface gráfica de apoio para o
PotgreSql. Para criar era necessário ja estar com o PgAdmin aberto e conectado
a um servidor de banco de dados que neste caso era em servidor local como pode
ser visto na Figura \ref{fig:desws}.

	\begin{figure}[h!]
		\centerline{\includegraphics[scale=1]{./imagens/2_q_metodologico/4_procedimentos_resultados/43_webservice/432_desenvolvimento/desws.png}}
		\caption[Servidor de banco de dados local no PgAdmin]{Servidor de banco de
		dados local no PgAdmin.
			\textbf{Fonte:}Elaborado pelos autores.}
		\label{fig:desws}
	\end{figure}
	
	\par Para a efetiva criação do banco de dados era necessário clicar com o
botão direito do \textit{mouse}, sobre a opção \textbf{Databases -> New
Database\ldots} no PgAdmin, apresentada na Figura \ref{fig:desws1}.

	\begin{figure}[h!]
		\centerline{\includegraphics[scale=0.8]{./imagens/2_q_metodologico/4_procedimentos_resultados/43_webservice/432_desenvolvimento/desws1.png}}
		\caption[Opção \textit{New Database\ldots}]{Opção \textit{New Database\ldots}.
			\textbf{Fonte:}Elaborado pelos autores.}
		\label{fig:desws1}
	\end{figure}

	\pagebreak
	
	\par Em seguida foi necessário preencher o dados da janela apresentada, como
está apresentado na Figura \ref{fig:desws2}.
	
	\begin{figure}[h!]
		\centerline{\includegraphics[scale=1]{./imagens/2_q_metodologico/4_procedimentos_resultados/43_webservice/432_desenvolvimento/desws2.png}}
		\caption[Tela \textit{New Database\ldots}]{Tela \textit{New Database\ldots}.
			\textbf{Fonte:}Elaborado pelos autores.}
		\label{fig:desws2}
	\end{figure}
	\pagebreak

	\par Como pode ser visto foram preenchidos os campos nome e usuário . O campo
nome se refere ao nome do banco de dados que foi definido com
\texttt{wsappunivas}, e usuário, o responsável pelo banco de dados, que para
este caso foi usuário padrão do SGDB, que é o \texttt{postgres}. Além destas
configurações mais nenhuma foi necessária. O banco de dados foi criado, porém
sua estrutura não foi definida, pois como será visto mais adiante o Hibernate,
possui um mecanismo, que com algumas configurações, permite a estruturação do
banco de dados, de acordo com o mapeamento objeto-relacional e de acordo com a
evolução do projeto. Isto permitirá mudanças na estrutura do banco de dados e
suas tabelas, e até mesmo eventuais correções.
	
	%02 - Início do projeto web no eclipse;
	\par Em seguida foi criado um projeto do tipo Dynamic Web Project no
Eclipse. Para proceder com a criação de um novo projeto deste tipo no Eclipse, é
necessário acessar na IDE, a opção \textbf{File -> New-> Dynamic Web Project}
como pode ser visto na figura \ref{fig:desws3}.

	
	\begin{figure}[h!]
		\centerline{\includegraphics[scale=0.8]{./imagens/2_q_metodologico/4_procedimentos_resultados/43_webservice/432_desenvolvimento/desws3.png}}
		\caption[Tela \textit{New Database\ldots}]{Tela \textit{New Database\ldots}.
			\textbf{Fonte:}Elaborado pelos autores.}
		\label{fig:desws3}
	\end{figure}
	
	\pagebreak
	
 	\par Em seguida foi apresentada uma tela para o preenchimento de alguns dados
 requeridos para a criação do projeto. Destas informações somente foi preenchido
 o nome do projeto. As outras informações continuaram sendo as que vem por
 padrão da IDE. A janela apresentada e as informações preenchidas podem ser
 vistas na Figura \ref{fig:desws4}.

	\begin{figure}[h!]
		\centerline{\includegraphics[scale=0.8]{./imagens/2_q_metodologico/4_procedimentos_resultados/43_webservice/432_desenvolvimento/desws4.png}}
		\caption[Tela para criação de um novo projeto no Eclipse]{Tela para criação de um novo projeto no Eclipse.
			\textbf{Fonte:}Elaborado pelos autores.}
		\label{fig:desws4}
	\end{figure}
	
	\pagebreak
	
	
	\par Na próxima janela apresentada, que têm por função configurar a pasta de
códigos do projeto manteve-se a configuração apresentada pela IDE, como mostra
a Figura \ref{fig:desws5}.

	\begin{figure}[h!]
		\centerline{\includegraphics[scale=0.8]{./imagens/2_q_metodologico/4_procedimentos_resultados/43_webservice/432_desenvolvimento/desws5.png}}
		\caption[Tela para criação de um novo projeto no Eclipse]{Tela para criação de um novo projeto no Eclipse.
			\textbf{Fonte:}Elaborado pelos autores.}
		\label{fig:desws5}
	\end{figure}
	
	\pagebreak
	
	\par Na sequencia, na tela que foi apresentada era necessário preencher o
campo \textbf{Context root:} com o contexto principal da aplicação web que
acabou mantendo o próprio nome da aplicação. Além disso foi marcado a opção
\textbf{Generate web.xml deployment descriptor}, para que ao criar o projeto, a
própria IDE criasse o arquivo \texttt{web.xml}, arquivo responsável por algumas
configurações da aplicação web. Esta tela esta apresentada na Figura
\ref{fig:desws6}.

	\begin{figure}[h!]
		\centerline{\includegraphics[scale=0.8]{./imagens/2_q_metodologico/4_procedimentos_resultados/43_webservice/432_desenvolvimento/desws6.png}}
		\caption[Tela para criação de um novo projeto no Eclipse]{Tela para criação de um novo projeto no Eclipse.
			\textbf{Fonte:}Elaborado pelos autores.}
		\label{fig:desws6}
	\end{figure}
	
	\pagebreak

	%03 - Mapeamento orm;	
		%	->Criação do pacote

	\par Após este passo foi concluído a criação do projeto, e já era possível
iniciar os trabalhos com a camada de persistência de dados do projeto. Antes de
começar o desenvolvimento foi necessário ainda criar uma pasta a qual foi a
responsável pos conter todos os arquivos \texttt{.jar} das bibliotecas que foram
usadas para o desenvolvimento do \textit{web service}. Para criar uma nova pasta dentro
do projeto foi necessário clicar com o botão direito do mouse sobre o projeto e
navegar sobre a opção \textbf{New -> Folder}, como pode ser visto na Figura
\ref{fig:desws6_1}.

	\begin{figure}[h!]
		\centerline{\includegraphics[scale=0.8]{./imagens/2_q_metodologico/4_procedimentos_resultados/43_webservice/432_desenvolvimento/desws6_1.png}}
		\caption[Tela para criação de um novo projeto no Eclipse]{Tela para criação de um novo projeto no Eclipse.
			\textbf{Fonte:}Elaborado pelos autores.}
		\label{fig:desws6_1}
	\end{figure}

	\pagebreak 
	
	\par Em seguida na janela apresentada, foi definido o nome da nova pasta como
\texttt{libs}. A tela e o nome preenchido podem ser vistos na Figura
\ref{fig:desws6_2}.

	\begin{figure}[h!]
		\centerline{\includegraphics[scale=0.60]{./imagens/2_q_metodologico/4_procedimentos_resultados/43_webservice/432_desenvolvimento/desws6_2.png}}
		\caption[Tela para criação de um novo projeto no Eclipse]{Tela para criação de um novo projeto no Eclipse.
			\textbf{Fonte:}Elaborado pelos autores.}
		\label{fig:desws6_2}
	\end{figure}

	\pagebreak
	
	\par Em seguida foram copiados todos os arquivos \texttt{.jar} da biblioteca
Hibernate que eram necessários ao projeto, para dentro desta pasta e também o
\texttt{jar} do \textit{driver} JDBC do PostGreSql, que seria responsável por
fazer a comunicação entre o banco de dados e a aplicação. Além disso era
necessário mais uma configuração para que as bibliotecas pudessem ser
reconhecidas como parte do projeto. Era necessário fazer a inclusão destas
bibliotecas para dentro do \textit{build path} do projeto. Para isso foi
necessário selecionar todos os arquivos \texttt{.jar} que estavam dentro da
pasta \texttt{libs}, clicar com o botão direito do mouse sobre eles e escolher
a opção \textbf{Build Path -> Add to Build Path}. A pasta \texttt{libs} que foi
criada, os arquivos \texttt{.jar} das bibliotecas usadas e a configuração do
\textit{build path}  do projeto, podem ser visto na Figura \ref{fig:desws6_3}.

	\begin{figure}[h!]
		\centerline{\includegraphics[scale=0.60]{./imagens/2_q_metodologico/4_procedimentos_resultados/43_webservice/432_desenvolvimento/desws6_3.png}}
		\caption[Tela para criação de um novo projeto no Eclipse]{Tela para criação de um novo projeto no Eclipse.
			\textbf{Fonte:}Elaborado pelos autores.}
		\label{fig:desws6_3}
	\end{figure}

	\par Com o projeto devidamente configurado, começou-se de fato a desenvolver a
camada de persistência da aplicação. Para este propósito, primeiramente foi
criado um pacote, onde ficaram contidas as classes que representam as entidades
do ORM. Para a criação do pacote foi necessário clicar com o botão direito do
mouse sobre o projeto e acessar a opção \textbf{New -> Package}, como pode ser
visto na Figura \ref{fig:desws7}.

	\begin{figure}[h!]
		\centerline{\includegraphics[scale=0.8]{./imagens/2_q_metodologico/4_procedimentos_resultados/43_webservice/432_desenvolvimento/desws7.png}}
		\caption[Tela para criação de um novo projeto no Eclipse]{Tela para criação de um novo projeto no Eclipse.
			\textbf{Fonte:}Elaborado pelos autores.}
		\label{fig:desws7}
	\end{figure}
	
	\pagebreak 
	
	\par Em seguida foi apresentada a janela New Java Package, para a criação de
um novo pacote mostrada na Figura \ref{fig:desws8}. O pacote recebeu o nome de
"\texttt{br.edu.univas.restapiappunivas.\\model}", pois nele estão contidas as
classes que fazem parte do modelo de negócios da aplicação. Este pacote foi
criado visando a divisão das responsabilidades internas no projeto, além de
contribuir positivamente com a organização do mesmo.

	\begin{figure}[h!]
		\centerline{\includegraphics[scale=0.8]{./imagens/2_q_metodologico/4_procedimentos_resultados/43_webservice/432_desenvolvimento/desws8.png}}
		\caption[Tela para criação de um novo projeto no Eclipse]{Tela para criação de um novo projeto no Eclipse.
			\textbf{Fonte:}Elaborado pelos autores.}
		\label{fig:desws8}
	\end{figure}
	
	\pagebreak
		
		%	->Criação das classes
	\par Com este pacote criado, ja era possível criar as classes do ORM. Foi
criada primeiramente a classe \texttt{Student.java}. Para a criação desta classe
foi necesário clicar com o botão direito do \textit{mouse} sobre o projeto e
navegar até a opção \textbf{New -> Class} como pode ser visto na Figura
\ref{fig:desws9}. 

	\begin{figure}[h!]
		\centerline{\includegraphics[scale=0.8]{./imagens/2_q_metodologico/4_procedimentos_resultados/43_webservice/432_desenvolvimento/desws9.png}}
		\caption[Opção para Criação de nova classe java no Eclipse]{Opção para
		Criação de nova classe java no Eclipse.
			\textbf{Fonte:}Elaborado pelos autores.}
		\label{fig:desws9}
	\end{figure}
	
	\pagebreak


	\par Em seguida foi apresentada uma janela chamada New Java Class. Nesta
janela somente foi necessário preencher o campo \textbf{Name:} que representa o
nome da classe que está sendo criada.
	
	
	
	 \begin{figure}[h!]
		\centerline{\includegraphics[scale=0.8]{./imagens/2_q_metodologico/4_procedimentos_resultados/43_webservice/432_desenvolvimento/desws10.png}}
		\caption[Tela para Criação de nova classe java no Eclipse]{Tela para Criação
		de nova classe java no Eclipse.
			\textbf{Fonte:}Elaborado pelos autores.}
		\label{fig:desws10}
	\end{figure}
	
	\pagebreak

	\par Esta classe foi definida para representar as informações referente aos
alunos. O código fonte desta classe pode ser visto na Figura \ref{fig:desws11}. 
	
	
	\begin{figure}[h!]
		\begin{lstlisting} [style=custom_Java]	
	package br.edu.univas.restapiapp.model;
	/**
	 *imports omitidos
	 */
	
	@Entity
	@Table(name = "student")
	public class Student {
	
		@Id
		@SequenceGenerator(name = "id_student", sequenceName = "seq_id_student",
			allocationSize = 1) 
		@GeneratedValue(generator = "id_student", strategy = GenerationType.IDENTITY)
		@Column(name = "id_student", nullable = false)
		private Long idStudent;
	
		@Column(name = "id_external", nullable = false)
		private Long idDatabaseExternal;
	
		@Column(length = 100, nullable = false)
		private String name;
	
		@Column(length = 100, nullable = false)
		private String email;
	
		@OneToMany(mappedBy="student", fetch = FetchType.EAGER)
		private List<Event> events;
	
		@OneToOne(optional = false, fetch = FetchType.LAZY)
		@JoinColumn(name = "id_user")
		private User user;
	
		/**
		 * Omitidos todos Getters e Setters
		 */
	
		@Override
		public int hashCode() {
			/**
			 * Omitido
			 */
		}
	
		@Override
		public boolean equals(Object obj) {
			/**
			 * Omitido
			 */
		}
	
	}
\end{lstlisting}
		\caption[Classe Student.java]{Classe \texttt{Student.java}.
			\textbf{Fonte:}Elaborado pelos autores.}
		\label{fig:desws11}
	\end{figure}
	
	\pagebreak
	
	%07 - Explicar anotações dos pojos
	\par É válido lembrar que esta classe possui anotações para que possa ser
reconhecida como uma entidade do JPA, e assim persistida no banco de dados
quando necessário. Além disso estas anotações possuem outras finalidades
específicas. A seguir estão listadas todas as anotações  que foram usadas na
classe \texttt{Student.java} e nas outras classes que fazem parte do mapeamento
objeto relacional da aplicação.

	\begin{itemize}
	  \item \texttt{@Entity}: esta anotação foi necessária para que esta classe
	  pudesse ser reconhecida como uma entidade do JPA e assim persistida no banco
	  de dados;
	  \item \texttt{@Table}: anotação que possui algumas configurações relativas a
	  tabela no banco de dados, a qual esta entidade representa, no caso da classe
	  mostrada anteriormente é configurado o nome da tabela;
	  \item \texttt{@Id}: esta anotação fica sobre o atributo que representa a
	  chave primária no banco de dados;
	  \item \texttt{@SequenceGenerator}: esta anotação define qual será o modo com
	  que a chave primaria será incrementada.
	  \item \texttt{@Column}: define algumas propriedades do campo da tabela do
	  banco de dados, o qual o atributo que ele anota representa. Estas
	  configuraçãoes podem são:
		  	\begin{itemize}
		    	\item \texttt{name}: muda o nome do campo;
		    	\item \texttt{length}: determina o tamanho em caracteres que o campo
		    	aceitará;
		    	\item \texttt{nullable}: define se o preenchimento do campo é obrigatório;
		    	\item \texttt{unique}: este atributo define se o campo aceitará valores
		    	únicos;
		    \end{itemize}
	  \item \texttt{@OneToMany}: representa um relacionamento um-para-muitos no
	  banco de dados. Anotam coleções de outras entidades;
	  \item \texttt{@ManyToOne}: representa um relacionamento
	  muitos-para-um no banco de dados. Este é a contraparte da anotação
	  um-para-muitos;
	  \item \texttt{@OneToOne}: representa um relacionamento um-para-um no banco de
	  dados.
\end{itemize}
 
	\par Esta classe faz parte do mecanismo de persistêcia de dados e é
simplesmente um  pojo ou seja, um objeto  simples que contêm somente atributos
privados e os métodos \textit{getters} e \textit{setters} que servem apenas
para encapsular estes atributos, e não possue nenhuma lógica de negócios. Além
desta classe, foram criadas outras com os mesmos propósitos. Estas classes
tinham a mesma finalidade da anterior, porém com pequenas diferenças no que diz
respeito à atributos, metodos e anotações. Estas classes representam, de
maneira individual, as tabelas no banco de dados. As classes podem ser vistas
na Figura \ref{fig:desws12}.
	
	
	\begin{figure}[h!]
		\centerline{\includegraphics[scale=0.8]{./imagens/2_q_metodologico/4_procedimentos_resultados/43_webservice/432_desenvolvimento/desws12.png}}
		\caption[Classes que fazem parte do ORM]{Classes que fazem parte do ORM.
			\textbf{Fonte:}Elaborado pelos autores.}
		\label{fig:desws12}
	\end{figure}
	
	\pagebreak

	%04 - HashCode e equals
	\par E por fim, para cada classe que representa uma entidade, foi necessário
implementar os métodos \texttt{hashCode} e \texttt{equals}, para que estas
pudessem facilmente ser comparadas e diferenciadas em relação aos seus valores,
haja visto que cada instância destas classes representa um registro no banco de
dados. A própria IDE provê uma forma facíl para criar este métodos, bastando
para isso clicar com o botão direito do mouse sobre o código da classe e
escolher a opção \textbf{Source -> Generate hashCode() and equals()\ldots} como
pode ser visto na Figura \ref{fig:desws13}.

	\begin{figure}[h!]
		\centerline{\includegraphics[scale=0.8]{./imagens/2_q_metodologico/4_procedimentos_resultados/43_webservice/432_desenvolvimento/desws13.png}}
		\caption[Opção Generate hashCode() and equals()\ldots]{Opção Generate
		hashCode() and equals()\ldots .
			\textbf{Fonte:}Elaborado pelos autores.}
		\label{fig:desws13}
	\end{figure}
	
	\pagebreak
	
	\par Em seguida na janela que foi apresentada foi necessário escolher qual
atributo seria usado para a comparação dentro dos métodos, como esta
apresentado na Figura \ref{fig:desws14} .
	
	\begin{figure}[h!]
		\centerline{\includegraphics[scale=0.8]{./imagens/2_q_metodologico/4_procedimentos_resultados/43_webservice/432_desenvolvimento/desws14.png}}
		\caption[Opção Generate hashCode() and equals()\ldots]{Opção Generate
		hashCode() and equals()\ldots .
			\textbf{Fonte:}Elaborado pelos autores.}
		\label{fig:desws14}
	\end{figure}
	
	\par Os métodos \texttt{hashCode} e \texttt{equals} da classe \texttt{Student.java}
foram implementados\\ usando o atributo \texttt{idStudent} como pode ser visto
na imagem \ref{fig:desws15}.

	\begin{figure}[h!]
		%implementação dos métodos hashCode() e equals()

\begin{lstlisting} [style=custom_Java] 	
	...
	
	@Override
	public int hashCode() {
		final int prime = 31;
		int result = 1;
		result = prime * result
				+ ((idStudent == null) ? 0 : idStudent.hashCode());
		return result;
	}

	@Override
	public boolean equals(Object obj) {
		if (this == obj)
			return true;
		if (obj == null)
			return false;
		if (getClass() != obj.getClass())
			return false;
		Student other = (Student) obj;
		if (idStudent == null) {
			if (other.idStudent != null)
				return false;
		} else if (!idStudent.equals(other.idStudent))
			return false;
		return true;
	}
	...
\end{lstlisting}

		\caption[Implementação os métodos hashCode() e equals()]{Implementação os
		métodos \texttt{hashCode()} e \texttt{equals()}.
			\textbf{Fonte:}Elaborado pelos autores.}
		\label{fig:desws15}
	\end{figure}
	
	\pagebreak
	
	
	\par Além destas classes, foi necessário criar um tipo enumerado (ou enum),
para definir quais seriam os tipos dos eventos, haja vista que estes teriam um
numero limitado de possibilidades. Para a criação desta, foi necesário clicar
com o botão direito do \textit{mouse} sobre o projeto e navegar até a opção
\textbf{New -> Enum} como pode ser visto na Figura \ref{fig:desws16}.
	
	\begin{figure}[h!]
		\centerline{\includegraphics[scale=0.8]{./imagens/2_q_metodologico/4_procedimentos_resultados/43_webservice/432_desenvolvimento/desws16.png}}
		\caption[Opção Generate hashCode() and equals()\ldots]{Opção Generate
		hashCode() and equals()\ldots .
			\textbf{Fonte:}Elaborado pelos autores.}
		\label{fig:desws16}
	\end{figure}

	\pagebreak
	
	\par Para esta enumeração foi definido o nome \texttt{EventType}. Os tipos de
eventos definidos foram três:

	\begin{itemize}
	  \item \texttt{PROVA\_AGENDADA}: que define um evento como agendamento de uma
	  atividade avaliativa;
	  \item \texttt{PROVA\_APLICADA}: que define um evento como, a efetiva
	  realização de uma atividade avaliativa;
	  \item \texttt{FALTAS}: representa o lançamento de uma falta;
	\end{itemize}
	
	\par A implementação da enumeração pode ser vista na Figura \ref{fig:desws15}.

	\begin{figure}[h!]
		%implementação dos métodos hashCode() e equals()

\begin{lstlisting} [style=custom_Java] 	

	package br.edu.univas.restapiappunivas.model;
	
	public enum EventType {
		PROVA_AGENDADA, PROVA_APLICADA, FALTAS
	}

\end{lstlisting}




		\caption[Implementação os métodos hashCode() e equals()]{Implementação os
		métodos \texttt{hashCode()} e \texttt{equals()}.
			\textbf{Fonte:}Elaborado pelos autores.}
		\label{fig:desws15}
	\end{figure}

	%05 - Configuração do persistence.xml
	\par Após a criação das entidades, foi necessário configurar o arquivo
\texttt{persistence.xml}. Foi necessário criar a pasta META-INF dentro da pasta
de de códigos fontes do projeto que também é conhecida como \textit{classpath},
com a finalidade de abrigar este arquivo. Em seguida foi criado o arquivo
dentro da pasta criada. Este arquivo é extremamente importante, pois é nele que
estão todas as configurações relativas à conexão com o banco de dados,
configurações referentes ao Dialeto SQL que vai ser usado para as consultas e
configurações referentes ao \textit{persistence unit} que é o conjunto de
classes mapeadas para o banco de dados. Este por sua vez recebeu o nome de
\texttt{WsAppUnivas}. Uma destas configurações merece uma atenção especial
trata-se da configuração \texttt{<property name="hibernate.hbm2ddl.auto"
value="create" />} que é a responsável por definir que o próprio Hibernate irá
criar a estrutura do banco de dados (tabelas, sequences, etc.) através do
mapeamento objeto relacional das classes. O arquivo \texttt{persistence.xml}
está exposto na Figura \ref{fig:desws16}.

	\begin{figure}[h!]
		\input{code/persistence_xml}
		\caption[Arquivo \texttt{persistence.xml}]{Arquivo \texttt{persistence.xml}.
		\textbf{Fonte:}Elaborado pelos autores.}
		\label{fig:desws16}
	\end{figure}
	
	\pagebreak
	
	%06 - Confecção JpaUtil.java
	\par Em seguida à confecção do \texttt{persistence.xml} foi criada a
classe \texttt{JpaUtil} que está representada na Figura \ref{fig:qm12}.
Esta classe é responsável por criar uma \texttt{EntityManagerFactory}. Este por
sua vez é uma  fábrica de instâncias de \texttt{EntityManager} que é um
\textit{persistence unit} ou unidade de persistência. Essa classe tem a
responsabilidade de prover um modo de comunicação entre a aplicação e o banco
de dados. No entanto a classe \texttt{JpaUtil} cria uma única instância de
\texttt{EntityManagerFactory}, que é responsável por disponibilizar e gerenciar
as instâncias de \texttt{EntityManager} de acordo com a necessidade da
aplicação.
	
	\begin{figure}[h!]
		%classe JpaUtil.java

\begin{lstlisting} [style=custom_Java] 	
package br.edu.univas.restapiappunivas.util;

/**
 *Imports Omitidos
 */

public class JpaUtil {
	private static EntityManagerFactory factory;

	static {
		factory = Persistence.createEntityManagerFactory("WsAppUnivas");
	}

	public static EntityManager getEntityManager() {
		return factory.createEntityManager();
	}

	public static void close() {
		factory.close();
	}

}
	
\end{lstlisting}
		\caption[Classe \texttt{JpaUtil.java}]{Classe \texttt{JpaUtil.java}.
		\textbf{Fonte:}Elaborado pelos autores.}
		\label{fig:qm12}
	\end{figure}

	\pagebreak
	
	\par Depois de finalizada a criação da camada de persistência do projeto foi
necessário uma configuração adicional. Percebeu-se que além das bibliotecas já
usadas seriam necessárias mais algumas bibliotecas para que se pudesse chegar
ao resultado final esperado. Por esse motivo tornava-se inviável ficar
controlando as bibliotecas de maneira manual no projeto. Foi necessário então,
fazer a conversão do projeto 















	%23 - Módulo que ira fazer a busca dos dados na base da instituição de ensino
	%24 - Falar que vai ser simulado
	\par Para que fosse possível transmitir dados para o aplicativo, era
necessário receber as informações do sistema acadêmico da referida instituição,
haja vista que o \textit{web service} é independente do mesmo. Para esse
propósito é necessário um módulo que faça a importação dos dados necessários
para a base de dados do \textit{web service}.

	\par Este por sua vez terá a responsabilidade de fazer a importação dos dados
periodicamente, e ainda tratar os tipos de dados recebidos para tipos
aplicáveis ao banco de dados local. Além disso é preciso notificar o módulo
responsável por invocar o serviço Google Cloud Messaging para que os
dispositivos dos alunos aos quais houveram atualizações nos dados, fossem
notificados e fizessem acesso ao \textit{web service} para solicitar esses
dados atualizados.

	\par Os procedimentos acima citados foram os passos até agora realizados com o
propósito de se alcançar os resultados esperados para essa pesquisa.






%08 - Finalizando camada de persistência
%09 - Camada de serviço
%10 - Classes que disponibilizam serviços anotações
%11 - Explicar as entities criadas para disponibilizar os dados
%12 - Ctrls que fazem a busca dos dados
%13 - Problema do erro 500
%14 - Provedor de arquivos e contexto
%15 - Em todos citar o pom.xml
%16 - Configuração do web.xml
%17 - Módulo de varredura de atualizações com timerTask
%18 - Módulo de alerta de provas agendas no dia da prova
%20 - Módulo para disparar as mensagens para o gcm
%21 - Serviço que faz o registro de sender_id
%22 - Mostrar a estrutura do empacotamento depois de finalizado