%\subsubsection{Desenvolvimento}

	\par Essas classes foram criadas fazendo uso de anotações próprias do
\textit{Hibernate}, que é um \textit{framework} que implementa a especificação
JPA\footnote{JPA - \textit{Java Persistense API}}. Essas classes fazem parte
dos mecanismos de persistêcia de dados e são simplesmente t ou seja, objetos
simples que contêm somente atributos privados e os métodos \textit{getters} e
\textit{setters} que servem apenas para encapsular estes atributos. Uma das
classes criadas, foi a classe \texttt{Aluno.java} que representa a tabela
\texttt{alunos} no banco de dados e está representada na Figura
\ref{fig:qm10}.%mudar para figura
	
		\begin{figure}[h!]
			\centerline{\includegraphics[scale=0.7]{./imagens/2_q_metodologico/qm10.png}}
			\caption[Classe \texttt{Aluno}]{Classe \texttt{Aluno}.
			\textbf{Fonte:}Elaborado pelos autores.}
			\label{fig:qm10} 
		\end{figure}
		
		\pagebreak
		
	\par Foram criadas outras classes \textit{Java} com a mesma finalidade da
anterior, porém com pequenas diferenças no que diz respeito à atributos,
metodos e anotações. Estas classes representam, de maneira individual, as
tabelas no banco de dados. Certos atributos dessas classes têm por finalidade
representar as colunas de cada tabela. Já os atributos que armazenam instâncias
de outras classes ou até mesmo conjuntos (coleções) de instâncias representam
os relacionamentos entre as tabelas. E por fim, para cada classe que representa
uma entidade, foi necessário implementar os métodos \texttt{hashCode} e
\texttt{equals}, para que estas pudessem facilmente ser comparadas e
diferenciadas em relação aos seus valores, haja visto que cada instância destas
classes representa um registro no banco de dados.
\overfullrule=2cm		
	\par Em seguida à criação das entidades, foi necessário configurar o arquivo
\texttt{persistence.\\xml} que fica dentro do \textit{classpath} do projeto
\textit{Java} ou seja, dentro da mesma pasta onde estão contidos pacotes do
projeto. Este arquivo é extremamente importante, pois é nele que estão todas as
configurações relativas à conexão com o banco de dados, configurações
referentes ao Dialeto SQL que vai ser usado para as consultas e configurações
referentes ao \textit{persistence unit} que é o conjunto de classes mapeadas
para o banco de dados.	O arquivo \texttt{persistence.xml} está exposto no
código \ref{fig:qm11}.
	
 		\begin{figure}[h!]
			\centerline{\includegraphics[scale=0.6]{./imagens/2_q_metodologico/qm11.png}}
			\caption[Arquivo \texttt{persistence.xml}]{Arquivo \texttt{persistence.xml}.
			\textbf{Fonte:}Elaborado pelos autores.}
			\label{fig:qm11}
		\end{figure}
		
	\par Em seguida à confecção do \texttt{persistence.xml} foi criada a classe
\texttt{JpaUtil} que está representada na Figura \ref{fig:qm12}.
Esta classe é responsável por criar uma \texttt{EntityManagerFactory} que é
uma  fábrica de instâncias de \texttt{EntityManager} que nada mais é que um
\textit{persistence unit} ou unidade de persistência. Essa classe tem a
responsabilidade de prover um modo de comunicação entre a aplicação e o banco
de dados. No entanto a classe \texttt{JpaUtil} cria uma única instância de
\texttt{EntityManagerFactory}, que é responsável por disponibilizar e
gerenciar as instâncias de \texttt{EntityManager} de acordo com a necessidade
da aplicação.
		
		\pagebreak
		\begin{figure}[h!]
			\centerline{\includegraphics[scale=0.7]{./imagens/2_q_metodologico/qm12.png}}
			\caption[Classe \texttt{JpaUtil}]{Classe \texttt{JpaUtil}.
			\textbf{Fonte:}Elaborado pelos autores.}
			\label{fig:qm12}
		\end{figure}
		
	\par Em seguida à construção das classes que fazem a parte da persistência de
dados, foi desenvolvido a parte de disponibilização de serviços
\textit{RESTful}, fazendo uso do \textit{framework} \textit{Jersey}. Com isso
pode-se construir a classe que representa o primeiro serviço do
\textit{webservice}, que é a classe \texttt{Alunos}. Essa classe representa um
contexto REST, e portanto, dispõe de alguns recursos. Esses recursos fazem a
recuperação e a transmissão dos dados do \textit{webservice} para o aplicativo
\textit{Android}. Essa classe e seus respectivos métodos  estão representada na
Figura \ref{fig:qm13}.

		\begin{figure}[h!]
			\centerline{\includegraphics[scale=0.7]{./imagens/2_q_metodologico/qm13.png}}
			\caption[Classe \texttt{AlunosService}]{Classe \texttt{AlunosService}.
			\textbf{Fonte:}Elaborado pelos autores.}
			\label{fig:qm13}
		\end{figure}
		
		\par O \textit{webservice} pode fazer a busca de alunos pelo \texttt{id}
passado ou retornar uma coleção de eventos vinculados a um alunos, dependendo
do recurso acessado. Os tipos de dados que o \textit{webservice} consome e
retorna é o JSON\footnote{JSON - \textit{Javascript Object Notation}}. Não foi
necessário fazer nenhuma implementação adicional relativa a este formato, pois
o próprio \textit{framework Jersey} faz o tratamento e a conversão dos tipos de
entrada e saída de dados. No caso do saída de dados, faz a conversão de objetos 
\textit{Java} para JSON. E no caso de entrada tranforma um JSON em objeto
\textit{Java} já conhecido pelo \textit{webservice}. Com isso concluiu-se o
desenvolvimento do \textit{webservice} que fornece os dados para o aplicativo.

	\par Para que fosse possível transmitir dados para o aplicativo, era
necessário receber as informações do sistema acadêmico da referida instituição,
haja vista que o \textit{web service} é independente do mesmo. Para esse
propósito é necessário  contruir um módulo que faça a importação dos dados
necessários para a base de dados do \textit{web service}. Este por sua vez
terá a responsabilidade de fazer a importação dos dados periodicamente, e
ainda tratar os tipos de dados recebidos para tipos aplicáveis ao banco de
dados local. Além disso é preciso notificar o módulo responsável por invocar
o serviço \textit{Google Cloud Messaging} para que os dispositivos dos alunos
aos quais houveram atualizações nos dados, fossem notificados e fizessem acesso
ao \textit{web service} para solicitar esses dados atualizados.

	\par Os procedimentos acima citados foram os passos até agora realizados com o
propósito de se alcançar os resultados esperados para essa pesquisa.