%\subsubsection{Montagem do Ambiente de Desenvolvimento}

	\par No que diz respeito à contrução do \textit{webservice}, foi necessária a
instalação e configuração de um ambiente de desenvolvimento compatível com as
necessidades apresentadas pelo \textit{software}. 

	\par A princípio foi instalado o \textit{Servlet Container Apache Tomcat} em
sua versão de número 7. Esse \textit{Servlet Container} foi instalado pois
implementa a API da especificação \textit{Servlets} 3.0 do \textit{Java}. Isso
era necessário pelo fato que o \textit{framework Jersey} usa \textit{servlets}
para disponibilizar serviços REST. Além disso o \textit{Apache Tomcat} foi
escolhido, para que o \textit{WebService} pudesse fornecer os serviços
necessários para o consumo, na arquitetura REST, que sugere o uso do protocolo
HTTP\footnote{HTTP - Hypertext Transfer Protocol} para troca de mensagens, pois
além da funcionalidade com \textit{Servlets}, o \textit{Apache Tomcat} também é
um servidor HTTP.
	
	\par O \textit{Apache Tomcat} foi instalado, por meio do \textit{download} de
um arquivo \texttt{.zip} no site oficial do mesmo. A instalação consiste apenas
em extrair os dados do arquivo em uma pasta da preferência do desenvolvedor.
Esta abordagem permitiu a integração do \textit{Apache Tomcat} com o
IDE\footnote{IDE - \textit{Integrated Development Environment}}
\textit{Eclipse}, além disso foi possível controlar e monitorar, o servidor de
aplicações através da IDE. Além da configuração necessária para integrar o
servidor à IDE, nem uma outra configuração foi necessária.

	%\par No ambiente de produção como está sendo usado um sistema operacional
%baseado em GNU/Linux, o mesmo foi instalado através do gerenciador de pacotes
%da distribuição. Em ambos os casos não foram necessárias configurações além do
%trivial de cada plataforma.
			
	\par Para armazenar os dados gerados e/ou recebidos, foi necessário fazer a
intalação do Sistema Gerenciador de Banco de Dados(SGBD) \textit{PostGreSql} na
sua versão de número {9.4}. Como está sendo usado um sistema operacional
baseado em GNU/Linux como ambiente de desenvolvimento, o mesmo foi instalado através do
gerenciador de pacotes da distribuição. Além disso foi necessário criar um
usuário no SGDB que tivesse permissão suficiente apenas para fazer as operações
referentes ao banco de daodos do \textit{web service}, evitando assim a
necessidade de se trabalhar diretamente com um usuário master do SGBD. 
%Essa
%medida foi tomada com intuito de prover maior segurança ao banco de dados do
%\textit{web service} , pois um usuário master tem permissão total de acesso a
%todos os bancos de dados gerenciados pelo SGBD.

%eclipse
	\par Eclipse ide
%maven 
	\par Maven plugin