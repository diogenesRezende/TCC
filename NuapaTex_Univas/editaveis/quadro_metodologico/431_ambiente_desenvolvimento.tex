%\subsubsection{Montagem do Ambiente de Desenvolvimento}

	\par No que diz respeito à contrução do \textit{webservice}, foi necessária a
instalação e configuração de um ambiente de desenvolvimento compatível com as
necessidades apresentadas pelo \textit{software}. 

	\par A princípio foi instalado o \textit{Servlet Container Apache Tomcat} em
sua versão de número 7. Esse \textit{Servlet Container} foi instalado pois
implementa a API da especificação \textit{Servlets} 3.0 do \textit{Java}. Isso
era necessário pelo fato que o \textit{framework Jersey} usa \textit{servlets}
para disponibilizar serviços REST. Além disso o \textit{Apache Tomcat} foi
escolhido, para que o \textit{WebService} pudesse fornecer os serviços
necessários para o consumo, na arquitetura REST, que sugere o uso do protocolo
HTTP\footnote{HTTP - Hypertext Transfer Protocol} para troca de mensagens, pois
além da funcionalidade com \textit{Servlets}, o \textit{Apache Tomcat} também é
um servidor HTTP.
	
	\par O \textit{Apache Tomcat} foi instalado no ambiente de desenvovimento, por
meio do \textit{download} de um pacote \texttt{.zip} no site oficial do mesmo.
Esta abordagem permitiu a integração do \textit{Apache Tomcat} com o
IDE\footnote{IDE - \textit{Integrated Development Environment}}
\textit{Eclipse}.

	\par No ambiente de produção como está sendo usado um sistema operacional
baseado em GNU/Linux, o mesmo foi instalado através do gerenciador de pacotes
da distribuição. Em ambos os casos não foram necessárias configurações além do
trivial de cada plataforma.
			
	\par Para armazenar os dados gerados e/ou recebidos, foi necessário fazer a
intalação do Sistema Gerenciador de Banco de Dados(SGBD) \textit{PostGreSql} na
sua versão de número 9.2.

%eclipse
	\par Eclipse ide 
%maven 
	\par Maven plugin