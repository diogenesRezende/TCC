%\subsubsection{Montagem do Ambiente de Desenvolvimento}
	
	\par No que diz respeito à contrução do \textit{web service}, foi necessária a
instalação e configuração de um ambiente de desenvolvimento compatível com as
necessidades apresentadas pelo software.

	\par A princípio foi instalado o Servlet Container Apache Tomcat em sua versão
de número {7}. Esse Servlet Container foi instalado pois implementa a API da
especificação Servlets {3.0} do Java. Isso era necessário pelo fato que o
\textit{framework} Jersey usa \textit{servlets} para disponibilizar serviços
REST. Além disso o Apache Tomcat foi escolhido, para que o \textit{web service}
pudesse fornecer os serviços necessários para o consumo, na arquitetura REST,
que sugere o uso do protocolo HTTP\footnote{HTTP - Hypertext Transfer Protocol}
para troca de mensagens, pois além da funcionalidade com Servlets, o Apache
Tomcat também é um servidor HTTP.
	
	\par O Apache Tomcat foi instalado, por meio do \textit{download} de um
arquivo compactado no site oficial do mesmo. A instalação consiste apenas em
extrair os dados do arquivo em uma pasta da preferência do desenvolvedor.
Esta abordagem permitiu a integração do Apache Tomcat com o
IDE\footnote{IDE - Integrated Development Environment}
Eclipse, que foi usada para o desenvolvimento. Com isto foi possível controlar
e monitorar, o servidor de aplicações através da IDE. Além da configuração
necessária para integrar o servidor à IDE, nenhuma outra configuração foi
necessária.

	\par Como ferramenta para desenvolvimento, foi usada a IDE Eclipse na versão
{4.4}, que é popularmente conhecida como Luna. O processo de instalação e
configuração da IDE, se assemelha bastante ao processo de instalação do Apache
Tomcat, pois somente é necessário fazer o download do arquivo compactado que é
fornecido na página do projeto, e descompactá-lo no local preterido pelo
desenvolvedor.

	\par Para armazenar os dados gerados e/ou recebidos, foi necessário fazer a
intalação do Sistema Gerenciador de Banco de Dados(SGBD) PostGreSql na sua
versão de número {9.4}. Como está sendo usado um sistema operacional baseado em
GNU/Linux como ambiente de desenvolvimento, o PostGreSql foi instalado através
do gerenciador de pacotes da distribuição.
 
%	\par Foi necessário criar um usuário no SGDB que tivesse permissão suficiente
%apenas para fazer as operações referentes ao banco de dados do \textit{web
%service}, evitando assim a necessidade de se trabalhar diretamente com um
%usuário master do SGBD. Esta medida foi tomada visando a segurança do banco de
%dados, pois com isto foi possível isolar e restringir as responsabilidades
% deste usuário.
