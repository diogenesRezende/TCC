%\chapter*{Introdução}
%\begin{center}
  \vspace{1.2em}
  \textbf{\large INTRODUÇÃO}
  \vspace{2.9em}
%\end{center}
\thispagestyle{empty}

\addcontentsline{toc}{chapter}{INTRODUÇÃO}
\stepcounter{chapter} %incrementa o número do capítulo

	\par Atualmente, com os avanços tecnológicos, as pessoas estão cada vez
mais conectadas e procuram soluções para seus problemas, que possam ajudá-las
de forma rápida e fácil.  Segundo \citeonline{lacheta2013}, tanto as empresas
quantos os desenvolvedores buscam plataformas modernas e ágeis para a criação
de aplicações. Esse fato contribuiu consideravelmente para o crescimento das
plataformas móveis de comunicação.

	\par Uma das áreas que mais se expandiu nos últimos anos é a de telefonia
móvel. \citeonline[p.1]{monteiro2012} afirma que “os telefones celulares
foram evoluindo, ganhando cada vez mais recursos e se tornando um item quase
indispensável na vida das pessoas”. Essa evolução no \textit{hardware}
possibilitou o crescimento, mobilidade e portabilidade do \textit{software}.

	\par Muitas coisas que antes eram feitas apenas em computadores
\textit{desktops} já podem ser realizadas nos celulares, como
transferências bancárias, localização de taxi, conversas entre amigos,
entretenimento com jogos e vídeos, entre outros. Ainda de acordo com
\citeonline{monteiro2012}, a plataforma \textit{Android} se destaca no
mercado devido ao grande número de aparelhos espalhados pelo mundo
além das facilidades que provêem aos desenvolvedores.

	\par Hoje em dia, há uma gama enorme de aplicativos para \textit{Android}
que tem como objetivo resolver problemas específicos. \citeonline{mendes2011},
vendo as dificuldades encontradas pelas novas bandas musicais em saber as opiniões
de seus fãs referente a \textit{shows} realizados, criou um aplicativo que tornou
possível a interação entre eles. \citeonline{oglio2013}, desenvolveu um utilitário
para \textit{Android} que possibilitou aos alunos do Centro Universitário Univates
acessarem o portal virtual de sua faculdade.


	\par Pensando-se nas facilidades providas pelos dispositivos móveis em
conseguir informações rápida a qualquer hora e local e visando facilitar o
acesso dos alunos as suas notas, faltas e provas agendadas, essa pesquisa
tem por finalidade o desenvolvimento de um aplicativo para a plataforma
\textit{Android} que possibilite aos alunos da Universidade do Vale do Sapucaí
receberem notificações e consultarem suas notas, faltas e provas agendadas.

%\chapter{OBJETIVOS}
	
	\par Para alcançar o propósito principal do trabalho,
o objetivo geral foi dividido em alguns objetivos específicos,
aos quais pode-se citar:
	
	\begin{itemize}
	  
	  \item Levantar requisitos do \textit{software} proposto de acordo com as
	  necessidades dos discentes.
	  
	  \item Desenvolver um aplicativo para dispositivos móveis na plataforma
	  \textit{Android}.
	  
	  \item Desenvolver um \textit{web service} para prover os dados necessários
	  para o bom funcionamento do aplicativo.
	
	\end{itemize}
	
	\par Desta maneira, esse trabalho contribui socialmente com os graduandos,
pois espera-se agilizar o processo em que eles consultam os resultados dos
exercícios avaliativos, faltas e as provas agendadas. O \textit{software}
também irá notificá-los no momento em que haver algum lançamento referente
as disciplinas cursadas, evitando assim que o estudante tenha que acessar o
portal do aluno várias vezes ao dia, ansioso em saber seu rendimento. 

	\par O projeto coopera na qualificação dos envolvidos do trabalho,
tendo em vista o aumento na procura por profissionais habilitados com
tecnologias atuais como \textit{Java}, \textit{Android}, REST,
\textit{Hibernate} entre outros.

	\par Para a universidade, essa pesquisa à coloca como pioneira nesse
quesito e demostra a sua preocupação com o bem estar de seus alunos,
pois existem pessoas que não tem computadores, mas possuem
\textit{smartphones}, com isso eles também conseguirão ter suas informações em tempo real.

	\par Na área acadêmica, esse trabalho contribui aos alunos do curso
de sistemas de informação servindo lhes como referência para pesquisas
futuras ou usá-lo para implementar funcionalidades a este projeto.
