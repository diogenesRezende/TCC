%\chapter*{Introdução}
%\begin{center}
  \vspace{1.2em}
  \textbf{\large INTRODUÇÃO}
  \vspace{2.9em}
%\end{center}
\thispagestyle{empty}

\addcontentsline{toc}{chapter}{INTRODUÇÃO}
\stepcounter{chapter} %incrementa o número do capítulo

	\par Atualmente, com os avanços tecnológicos, as pessoas estão cada vez mais
conectadas e procuram soluções para seus problemas, que possam ajudá-las de
forma rápida e fácil. Segundo \citeonline{lacheta2013}, tanto as empresas
quantos os desenvolvedores buscam plataformas modernas e ágeis para a criação
de aplicações. Esse fato contruibuiu consideravelmente para o crescimento das
plataformas móveis de comunicação.

	\par Uma das áreas que mais se expandiu nos últimos anos é a de telefonia
móvel. \citeonline[p.1]{monteiro2012} afirma que “os telefones celulares
foram evoluindo, ganhando cada vez mais recursos e se tornando um item quase
indispensável na vida das pessoas”. Essa evolução no \textit{hardware}
possibilitou o crescimento, mobilidade e portabilidade do \textit{software}.

	\par Muito das coisas que antes eram feitas somente em computadores
\textit{desktops} já podem ser realizadas nos celulares, como transferências
bancarias, localização de taxi, conversas com amigos, entretenimento com jogos
e vídeos, entre outros.

	\par Ainda de acordo com \citeonline{monteiro2012}, a plataforma
\textit{Android} se destaca no mercado devido ao grande número de aparelhos
espalhados pelo mundo e pela facilidade que provêem aos desenvolvedores. Esta
plataforma foi utilizada para o desenvolvimento de vários trabalhos de
conclusão de curso como por exemplo \citeonline{mendes2011}, que criou um
aplicativo para que as bandas musicais pudessem ter mais interação com seus
fãs. \citeonline{oglio2013} do Centro Universitário Univates, criou um sistema
que permite o acesso ao portal virtual da sua faculdade.

	\par Pelas facilidades que \textit{smartphones} provêem para conseguir
informações rápidas a qualquer hora e local, pensa-se em criar um utilitário
que possibilite aos usuários consultarem as suas notas, presenças e provas
agendadas no portal do aluno.

%\chapter{OBJETIVOS}
	
	\par O obejetivo principal desta pesquisa é desenvolver um aplicativo,
para dispositivos moveis, na plataforma \textit{Android}, que permita aos
alunos da Universidade do Vale do Sapucaí consultarem suas notas, presenças e
provas agendadas. Porém o objetivo principal pode ser desmembrado em objetivos
menores e mais concisos, com a finalidade de conseguir realizá-lo com maior
eficácia. Estes, por sua vez são:
	
	\begin{itemize}
	  
	  \item Levantar requisitos do \textit{software} proposto de acordo com as
	  necessidades dos discentes.
	  
	  \item Desenvolver o aplicativo para dispositivos móveis na plataforma
	  \textit{Android}.
	  
	  \item Desenvolver um \textit{web service} para prover os dados necessários ao
	  aplicativo proposto.
	
	\end{itemize}
	
	\par Com esses passos espera-se fazer um \textit{software} eficaz que auxiliará
no dia-a-dia dos alunos.A escolha por fazer um aplicativo se deu pela
necessidade em acessar o portal do aluno para ter informações referente às
disciplinas. Com o projeto espera-se contribuir socialmente facilitando o
acesso dos usuários às suas notas, provas agendadas e faltas.

	\par O resultado final desta pesquisa também auxiliará os alunos do curso de
Sistemas de Informação que necessitarem saber como se desenvolve um aplicativo
na plataforma \textit{Android} ou implementar mais funcionalidades nesse
projeto.

	\par A plataforma \textit{Android} será utilizada devido a grande popularidade
do sistema operacional. Visando facilitar as pesquisas aos conteúdos publicados
no portal do alunos, quer-se desenvolver um App\footnote{Abreviação para a
palavra \textit{Application}}, pela qual os discentes os terão facilmente, pois
serão notificados quando houver alguma mudança, como por exemplo ao ser lançada
uma nota.
