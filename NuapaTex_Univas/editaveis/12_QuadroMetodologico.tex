\chapter{QUADRO METODOLÓGICO}

	\par Nesse capítulo serão apresentados os métodos adotados para se realizar a
pesquisa, tais como tipo de pesquisa, contexto, participantes, entre outros.

\section{Tipo de pesquisa}

	\par Pesquisa é ato de buscar e procurar pela resposta de algo.
\citeonline[p.15]{markoni2002} definem pesquisa como “uma indagação minuciosa
ou exame crítico e exaustivo na procura de fatos e princípios”.

	\par Existem diversos tipos de pesquisa, porém para se obter o verdadeiro
objetivo desta, será utilizada a pesquisa aplicada, ou seja, será desenvolvido
um projeto real que poderá ser utilizado por qualquer instituição de ensino.

	\par Segundo \citeonline[p.20]{markoni2002}, uma pesquisa do tipo aplicada
“caracteriza-se por seu interesse prático, isto é, que os resultados sejam
aplicados ou utilizados, imediatamente, na solução de problemas que ocorrem na
realidade”.

	\par Desta maneira, percebe-se que o projeto enquadra-se no tipo de pesquisa
aplicada, pois resolverá um problema específico, e para isso será criado uma
aplicativo para dispositivos móveis que facilitará aos graduandos acessarem o
portal do aluno de uma universidade.

\section{Contexto de pesquisa}

	\par Essa pesquisa será benéfica a uma instituição educacional que possua
um portal \textit{online}, pois facilitará o acesso dos discentes às suas
informações escolares. O objetivo é criar um aplicativo para dispositivos
móveis, porém inicialmente apenas para a plataforma Android, o qual notificará
os usuários quando houver alguma mudança, como por exemplo, ao ser lançada uma
nota.
	
	\par O aluno irá acessar o aplicativo com o mesmo \textit{login} do sistema
\textit{web} que acessará o \textit{web service}, este por sua vez será
responsável por buscar as informações no banco de dados e apresentá-las no
\textit{smartphone} do aluno. 
	
	\par Pretende-se conseguir acesso ao banco de dados do portal do aluno da
Univás a fim de realizar testes e, possivelmente, implantar o \textit{software}.


\section{Participantes}

	\par Os participantes serão responsáveis por planejar, executar e testar o
\textit{software} a ser desenvolvido. A seguir serão descritos os integrantes
que desenvolverão esta pesquisa.

	\par Diego D’leon Nunes, técnico em informática formado pelo INPETTECC, é aluno
do VII período do curso de sistemas de informação da universidade do vale do
sapucaí. Atualmente desempenha a função de analista de suporte na empresa
Automação e cia.

	\par Diógenes Aparecido Rezende, é aluno do VII período do curso de sistemas de
informação da universidade do vale do sapucaí. Atualmente desempenha a função de 
analista de suporte técnico na empresa NGTec Soluções em tecnologia LTDA.

	\par Henrique Almeida Versiani Murta, é aluno do VII período do curso de
sistemas de informação da universidade do vale do sapucaí. Atualmente
desempenha a função de assistente administrativo em compras na empresa A
Construtora Pouso Alegre LTDA.

	\par Professor Roberto Ribeiro Rocha, graduado em Ciência da Computação pela
faculdade de Administração e Informática – FAI (2002), possui especialização em
Produção de Software Livre pela Universidade Federal de Lavras – UFLA (2006) e
mestrado em Ciência e Tecnologia da Informação na Universidade Federal de
Itajubá – UNIFEI (2013). Foi analista de sistemas na Liveware Tecnologia a
Serviço Ltda e integrante da equipe de TI na Megatron Fios e Cabos Especiais.
Possui experiência na área de ciência da computação, com ênfase em arquitetura
de sistemas de computação, \textit{software} livre e Linux. Atualmente é
professor no curso de Sistemas de Informação na Univás.
	
\section{Instrumentos}

	\par Serão levantados requisitos e informações necessárias para o
desenvolvimento do projeto. Esses requisitos serão encontrados por meio de
de questionários, reuniões e pesquisas.

	\par Um questionário é uma forma de coletar informações através de algumas
perguntas feitas a um público específico. Segundo \citeonline{gunther2003},
questionário pode ser definido como um conjunto de perguntas que mede a opinião
e interesse de quem o responde.

	\par Devido a isto, será realizado um questionário com algumas perguntas,
respondidas pelos alunos da Univás. Esse questionário será feito de forma
informal e enviado por \textit{emails} e redes sociais. Com suas respostas,
pode-se saber a satisfação dos alunos com o portal, qual são suas maiores
dificuldade ao acessá-lo e a sua opinião caso houvesse um aplicativo que os
conectasse ao sistema da universidade.

	\par Reunião é unir-se com uma ou mais pessoas em um local, físico ou
remotamente para tratar algum assunto específico. De acordo com
\citeonline{celestino2013}, a reunião pode ser chamada de um momento
\textit{Kick-off}, que na área tecnológica é o momento em que os integrantes do
projeto se reúnem para definir objetivos, recursos e restrições ao projeto.
Normalmente esses encontros são realizados em local fora do ambiente de
trabalho. Serão realizadas reuniões entre os participantes afim de discutir o
andamento da pesquisa. Esses encontros tem por objetivo debater e traçar metas
para se chegar a solução.

	\par As pesquisas sobre o tema proposto serão feitas através de livros,
manuais, revistas e web sites.
	
\section{Procedimentos}

	\par Para se construir a aplicação proposta, é necessário executar alguns 
procedimentos para que se possa cumprir o objetivo final, que na maioria das
vezes, é o desenvolvimento de um \textit{software}. No caso desta pesquisa,
para que se cumpra o que já foi citado nos objetivos, são necessários os
seguintes procedimentos:

\begin{itemize}

	\item Fazer levantamento dos requisitos funcionais tanto do aplicativo na
	plataforma Android quanto do \textit{Web Service};

	\item Fazer a modelagem do sistema para que se possa garantir um nível de
	qualidade alto com um custo relativamente baixo;

	\item Construir as aplicações já citadas nos objetivos dessa pesquisa;

	\item Fazer a implantação das aplicações;

	\item Fazer testes para garantir a qualidade das mesmas.

\end{itemize}

	\par Com estes passos, se seguidos com extremo rigor, pretende-se alcançar os
objetivos propostos nessa pesquiza.


\pagebreak
\section{Cronograma}
\begin{table} [h]
  \caption[Cronograma de Atividades]
          {Cronograma de Atividades}
  \centering
  \begin{small}
  \setlength{\tabcolsep}{1pt} 
  \begin{tabular}{|p{3.2in}|c|c|c|c|c|c|c|c|c|c|c|c|}
    \hline
     
    \multicolumn{1}{|c|}{\textbf{Ações}}&
    \multicolumn{1}{ c|}{\textbf{Jan}}  & 
    \multicolumn{1}{ c|}{\textbf{Fev}}  &
    \multicolumn{1}{ c|}{\textbf{Mar}}  &
    \multicolumn{1}{ c|}{\textbf{Abr}}  &
    \multicolumn{1}{ c|}{\textbf{Mai}}  &
    \multicolumn{1}{ c|}{\textbf{Jun}}  &
    \multicolumn{1}{ c|}{\textbf{Jul}}  &
    \multicolumn{1}{ c|}{\textbf{Ago}}  &
    \multicolumn{1}{ c|}{\textbf{Set}}  &
    \multicolumn{1}{ c|}{\textbf{Out}}  &
    \multicolumn{1}{ c|}{\textbf{Nov}}  &
    \multicolumn{1}{ c|}{\textbf{Dez}}  \\\hline\hline 
    
    Pesquisa e levantamento de dados      &X&X&&&&&&&&&&\\\hline
    
    Escrita pré-projeto                   &X&X&&&&&&&&&&\\\hline
    
    Levantamento de requisitos funcionais do \textit{software}
    &&X&&&&&&&&&&\\\hline
    
    Escrita do quadro teórico   &&X&X&&&&&&&&&\\\hline
    
    Entrega do quadro teórico   &&&X&&&&&&&&&\\\hline
    
    Escrita do quadro metodológico      &&&X&X&&&&&&&&\\\hline
    
    Entrega do quadro metodológico      &&&&X&&&&&&&&\\\hline 
    
    Apresentação do projeto para a banca de qualificação &&&&&X&&&&&&&\\\hline
    
    Correções textuais para o TCC &&&&&X&X&&&&&&\\\hline
    
    Início do desenvolvimento do \textit{software} &&&&&&X&&&&&&\\\hline
    
    Escrita do TCC &&&&&&X&X&X&X&&&\\\hline
    
    Desenvolvimento do \textit{software} proposto &&&&&&X&X&X&X&&&\\\hline
    
    Apresentação do projeto para pré-banca &&&&&&&&&&X&&\\\hline
    
    Correções textuais do TCC &&&&&&&&&&X&X&\\\hline
    
    Apresentação do TCC para banca e para o público &&&&&&&&&&&X&\\\hline
    
    Correções textuais finais do TCC &&&&&&&&&&&X&X\\\hline
    
    Entrega do TCC &&&&&&&&&&&&X\\\hline
  
  \end{tabular}
  
  \end{small}
  
\end{table}

\pagebreak
\section{Orçamento}
\begin{table} [h]
  \caption[Orçamento]
          {Orçamento}
  \centering
  \begin{small}
  \setlength{\tabcolsep}{1pt}
  \begin{tabular}{|p{3.5in}|c|c|c|}
    \hline 
    \multicolumn{1}{|c|}{\textbf{Material}} & 
    \multicolumn{1}{c|}{\textbf{Quantidade}} &
    \multicolumn{1}{c|}{\textbf{Valor Unitário}} &
    \multicolumn{1}{c|}{\textbf{Total}}\\\hline\hline 
    
    Livro Google Android-Casa do Código &1 &R\$ 69,90 &R\$ 69,90 \\\hline
    
    Livro REST-Casa do Código &1 &R\$ 69,90 &R\$ 69,90 \\\hline
    
    Livro Java Web Services: Up and Running   &1   &R\$ 128,50 &R\$ 128,50 \\\hline
    
    Livro Web Services With Java   &1   &R\$ 103,50 &R\$ 103,50 \\\hline
    
    Livro SOA aplicado - Casa do Codigo  &1   &R\$ 69,90 &R\$ 69,90 \\\hline 
    
    Encadernações &10   &R\$ 1,50 &R\$ 15,00 \\\hline
    
    Impressões            &1200  &R\$ 0,25 &R\$ 300,00 \\\hline
    
    Xerox &1500  &R\$ 0,15 &R\$ 225,00 \\\hline
    
    Tranportes referentes à pesquisa &1  &R\$ 200,00 &R\$ 200,00 \\\hline
    
    Serviços de Terceiros &1  &R\$ 400,00 &R\$ 400,00 \\\hline
    
    Outros &1  &R\$ 200,00 &R\$ 200,00 \\\hline
    
    \textbf{Total}        &-   &-         &\textbf{R\$ 1781,70} \\\hline
    
  \end{tabular}
  \end{small}
  
\end{table}