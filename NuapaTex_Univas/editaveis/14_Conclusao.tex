
\chapter{CONCLUSÃO} 

	\par Com a realização desta pesquisa, foi criado um \textit{web service}, que
responde à requisições via REST, oferecendo a Univás uma oportunidade de
disponibilizar suas informações acadêmicas através de serviços. No presente
momento, o serviço possível de se utilizar é a consulta de notas, faltas e provas
agendadas.

	\par Além disso foi construído um aplicativo, que tem por finalidade
exemplificar uma possibilidade de uso do serviço criado. Além do mais, no
momento em que algum professor lançar alguma informação acadêmica referente ao
aluno, o aplicativo do mesmo deve ser notificado.

	\par As mensagens enviadas do servidor para o dispositivos móveis são
transmitidas através da API Google Cloud Messaging (GCM), da Google, que
oferece o recurso gratuitamente e que mostrou-se muito eficaz, solucionando o
problema do envio de notificações aos dispositivos dos alunos  e de transmissão
dos dados.

	\par Deve-se também destacar o grande número de materiais disponibilizados por
desenvolvedores e estudiosos da área, os quais possibilitaram aos autores desta
pesquisa o estudo e aprendizagem das teorias apresentadas nesta pesquisa, bem
como suas implementações.

	\par Portanto, apesar das dificuldades encontradas para a realização deste
trabalho, como realizar a comunicação entre o web service e o aplicativo
Android, percebeu-se que o software apresentado nesta pesquisa é de grande
utilidade aos alunos da Universidade do Vale do Sapucaí, pois conseguem
acompanhar seus desempenhos escolares através de seus equipamentos
\textit{mobile} e a Univás que tem agora, uma estrutura pronta para
disponibilizar suas informações através de serviços.

	\par Devido ao crescente número de dispositivos móveis é possível perceber uma
época favorável para explorar essas tecnologias. Sendo assim, este projeto
possibilita aos graduandos em Sistemas de Informação uma oportunidade para
acrescentar novas funcionalidades para esta aplicação em trabalhos futuros.

	\par Devido ao tempo escasso esse trabalho não trata a parte de segurança,
podendo ser implementado em outra oportunidade. São exibidas, apenas as
disciplinas do semestre corrente, sendo possível acrescentar a funcionalidade
para exibir todas as matérias já cursadas. Também pode-se criar o serviço para
que os alunos realizem a CPA, consultas de livros da biblioteca, permitir aos
professores lançarem notas no portal do aluno ou publicarem materiais, além de
possibilitar o acesso a outras plataformas como Windows Phone e IOS.

	\par Por fim, pode-se afirmar que o presente trabalho realizou seus objetivos,
os quais eram, desenvolver uma estrutura para a universidade poder
disponibilizar informações em forma de serviço, o que hoje ainda não acontece,
e possibilitar aos alunos da Univás consultarem suas notas, faltas e provas
agendadas, notificando-os quando estes eventos ocorrem. Ainda, esta
pesquisa foi de grande relevância aos participantes do projeto, pois contribuiu
por uma ampla visão de resolução de problemas e um conhecimento vasto nas
tecnologias utilizadas.
