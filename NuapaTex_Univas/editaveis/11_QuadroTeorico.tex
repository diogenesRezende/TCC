\chapter{QUADRO TEÓRICO}

	\par Neste capítulo serão descritos os principais conceitos e características
das tecnologias a serem utilizadas para o desenvolvimento dos \textit{softwares}
propostos nos objetivos dessa pesquisa.

\section{\textit{Java}}

	\par Segundo \citeonline{caelum1}, a \textit{Sun Microsystems}, no ano de 1992,
criou um time sob liderança de James Gosling para desenvolver inovações
tecnológicas, para rodar em pequenos dispositivos.

	\par De acordo com \citeonline{easyJavaMagazine2015}, como os recursos dos
equipamentos eram mínimos, o gestor tentou modificar e estender as linguagens
C/C++, porém acabou mudando de ideia e criando uma nova linguagem que foi
chamada de Oak, em português Carvalho, árvores existentes em frente ao seu
escritório.

	\par Contudo, os advogados responsáveis pelo registro da linguagem não
aprovaram a denominação escolhida. Após alguns debates, levantaram possíveis
nomes entre eles, \textit{Java} (gíria americana para café), pois sua equipe se
motivava com esse alimento.

	\par Entre as principais características pode–se dizer que o \textit{Java} é:
	\begin{itemize}
	  \item Orientado a objeto;
	  \item Seguro;
	  \item Independente de plataforma.  
	\end{itemize}
	
	\par \citeonline{romanato2015} afirma, que o Java usa a JMV (Java Virtual
Machine), que é uma máquina virtual capaz de converter os Byte Codes para a
linguagem do sistema operacional utilizado pelo cliente, sem a necessidade de
compila–lo para cada plataforma.Dessa maneira um software que é executado no
Windows, funcionará normalmente em qualquer outro sistema.

\section{\textit{Android}}

	\par Segundo Monteiro (2012), Android é um sistema operacional baseado no
Lixux, feito especialmente para dispositivos móveis, o qual começou a ser
desenvolvido no ano de 2003 pela então empresa Android Inc, que em 2005 foi
agregada ao Google.

	\par Por ser um software de código aberto, aliás, o primeiro para equipamentos
mobile, Krazit (2009) publicou uma entrevista com Rubin, um dos idealizadores
do Android, o qual afirma que o sistema pode rodar em equipamentos de diversos
fabricantes, evitando assim ficar limitado a poucos dispositivos. Conforme
informações do site Android (2015), hoje em dia existe mais de um bilhão de
aparelhos espalhados pelo mundo com esse sistema operacional.
	
	\par Conforme afirma Monteiro (2012) as aplicações são executadas em uma
máquina virtual denominada Dalvik. Cada utilitário, usa uma instancia dessa
máquina virtual tornando assim mais segura a aplicação, por outro lado os
softwares só podem acessar algum recurso do dispositivo caso seja formalmente
aceito pelo usuário ao instala-lo.

	\par Um dos recursos muito utilizado no Andriod são as chamadas Intents, que de
acordo com K19 (2015, p.29), “são objetos responsáveis por passar informações,
como se fossem mensagens, para os principais componentes da API do Android,
como as Activities, Services e BroadCast Receivers.” \par Monteiro (2012) diz
que as Intents são criadas quando se tem a intenção de realizar algo como por
exemplo compartilhar uma imagem, utilizando os recursos já existentes no
dispositivo. Existem dois tipos de Intents:

	\begin{itemize}
  		\item Intents implícitas – Quando não é informada qual atividade deve ser
 			chamada, ficando assim por conta do sistema operacional verificar qual a
 			melhor opção.
  		\item Intents explicitas – Quando é informada qual atividade deve ser
  			chamada. Usada normalmente para chamar activities da mesma aplicação.
	\end{itemize}
	
	\par Segundo K19 (2015) uma aplicação Android pode ser construída com quatro
tipos de componentes:

	\begin{itemize}
	  \item Activitie – São as telas com interface gráficas capazes de terem
	  		interações com os usuários.
	  \item Services – São serviços executados em segundo plano, com tarefas que
	  		levam algum tempo sem comprometer a interação do usuário.
	  \item Content providers – São os provedores de conteúdo que permitem o acesso e
			a modificação de dados.
	\item Broadcast receivers – São componentes capazes de responder a eventos do
			sistema operacional.
	\end{itemize}

	\par Com a ideia de desenvolver um aplicativo para dispositivos móveis, a
plataforma Android foi escolhida devido ao seu destaque no mercado, pela
facilidade que apresenta aos usuários e por ser um sistema operacional de
código livre.

\section{\textit{Android Studio}}

	\par Umas das ferramentas mais utilizadas para o desenvolvimento em Android é o
Eclipse IDE, contudo a Google criou um software especialmente para esse
ambiente, chamado Android Studio. Segundo Gusmão (2014), ele é uma IDE baseado
no ItelliJ Idea e foi apresentado na Conferência para desenvolvedores I/O de
2013.

	\par Carvalhos (2013) afirma que as maiores vantagens de se utilizar esse
programa é a possibilidade de customizar o tema e os atalhos. A programação
tornou-se mais rápida com o auto – complete que não necessita de nenhum
comando, pois ele já vai completando de acordo com que é feita a digitação.
Além disso, tem uma interface atraente com grande facilidade para a
programação, sendo que é possível arrastar os elementos da view. Existe também
uma forma simples para a integração com as ferramentas de controle de versão,
como o GitHub. Os downloads necessários para um projeto podem ser feitos
diretamente pela própria IDE sem ter a necessidade de ficar procurando nas
páginas dos desenvolvedores.

	\par Gusmão (2014) diz que a plataforma está disponível para Windows, Mac e
Linux, e os programadores terão disponíveis uma versão estável e mais três
versões que estarão em teste chamadas de Beta, Dev e Canary.


\section{\textit{Web Services}}
	
	\par Nos tempos atuais, com o grande fluxo de informação que percorre pelas
redes da \textit{internet} é necessário um nível muito alto de integração entre
as diversas plataformas, tecnologias e sistemas. Como uma provável solução para
esse ponto, já existem as tecnologias de sistemas distribuídos. Porém essas
tecnologias sofrem demasiadamente com o alto acoplamento de seus componentes e
também com a grande dependência de uma plataforma para que possam funcionar. Com
intuito de solucionar a estes problemas e proporcionar alta transparência entre
as várias plataformas, foram criados as tecnologias \textit{web services}.
	
	\par De acordo com \citeonline{erl2015}:
	\begin{citacao}
		No ano de 2000, a W3C (\textit{World Wide Web Consortium}) aceitou a submissão
		do \textit{Simple Object Access Protocol} (SOAP). Este formato de mensagem
		baseado em XML estabeleceu uma estrutura de transmissão para comunicação entre
		aplicações (ou entre serviços) via HTTP. Sendo uma tecnologia não amarrada a
		fornecedor, o SOAP disponibilizou uma alternativa atrativa em relação aos
		protocolos proprietários tradicionais, tais como CORBA e DCOM.
	\end{citacao}
	
	\par Considera-se então a existência dos \textit{web services} a partir daí. De
acordo com \citeonline{duraes2005}, \textit{Web Service} é um componente que
tem por finalidade integrar serviços distintos. O que faz com que ele se torne
melhor que seus concorrentes é a padronização do XML(\textit{Extensible Markup
Language}) para as trocas de informações. A aplicação consegue conversar com o
servidor através do WSDL que é documento que contém as regras de funcionamento
do \textit{web service}.
	
	\par Os \textit{web services} além de fornecerem uma padronização de
comunicação entre as várias tecnologias existentes, provê transparência na
troca de informações. Isso contribui pelo fato de que as novas aplicações
possam se comunicar com aplicações mais antigas ou aplicações contruídas sobre
outras plataformas.

	\par Além das tecnologias \textit{web services} tradicionais, existe também os
\textit{web services} REST que também disponibilizam serviços, porém não
necessitam de encapsulamento de suas mensagens assim como os \textit{web
Services} SOAP. Este fato influencia diretamente na performance da aplicação
como um todo, haja vista que não sendo necessário o encapsulamento da informação
requisitada ao \textit{web service}, somente é necessário o processamento e
tráfego da informação que realmente importa. As caracteristícas do padrão REST
serão abordadas a seguir.

	\subsection{REST}
	
	\par Segundo \citeonline{saudate2012}, REST é a sigla de
\textit{Representational State Transfer}, ou em português Transferência de
Estado Representativo, desenvolvido por Roy Fielding na defesa de sua tese de
doutorado. Segundo o próprio \citeonline{fielding2000} REST é um estilo que
deriva do vários estilos arquitectónicos baseados em rede e  que combinado
com algumas restrições, fornecem uma interface simples e uniforme para
fornecimento de serviços\footnote{Tradução e resumo de informações de
responsabilidade do autor da pesquisa.}.
	
	\par \citeonline{rubbo2015}, afirma que os dados e funcionalidade de um sistema
são considerados recursos e podem ser acessados através das URI's
\textit{(Universal Resource Identifier)}, facilitando dessa forma a comunicação
do servidor com o cliente.
	 	
	\par \citeonline{saudate2012}, explica ainda que os métodos do HTTP\footnote{
 HTTP – Abreviação para \textit{HyperText Transfer Protocol}} podem fazer
 modificações nos recursos através dos comandos:
	 
	 \begin{itemize}
	   \item GET – Para recuperar algum dado. 
	   \item POST – Para criar algum dado.
	   \item PUT – Para alterar algum dado. 
	   \item DELETE – Para excluir algum dado. 
	 \end{itemize}
	 
	\par Segundo \citeonline{godinho2009}, não há um padrão de formato para as
 trocas de informações, mas as que mais são utilizadas é o XML\footnote{XML –
 Abreviação para \textit{Extensible Markup Language}.} e o JSON\footnote{JSON –
 Abreviação para \textit{JavaScript Object Notation}.}. O REST\footnote{REST –
 Abreviação para \textit{Representational State Transfer}.} é o mais indicado
 para aplicações em dispositivos moveis, devido sua agilidade.
	
	
\section{\textit{Apache Tomcat}}

	\par De acordo com a \citeonline{tomcat2015}, \textit{Apache Tomcat} é uma
implementação de código aberto das especificações \textit{Java Servlet} e
\textit{JavaServer Pages}. O \textit{Apache Tomcat}  é um \textit{Servlet
Container}, que disponibiliza serviços através de requisições e respostas.
	
	\par Segundo \citeonline{tomcat2015}, o projeto desse \textit{software} começou
com a \textit{Sun Microsystems}, que em 1999 doou a base do código para
\textit{Apache Software Foundation} onde foi lançada a versão 3.0.
	
	\par Desta forma, o cliente envia uma requisição através do seu navegador, o
servidor por sua vez a recebe, executa o \textit{servlet} e devolve a resposta
ao usuário.

\section{PostgreSQL}

	\par De acordo com \citeonline{sourceforge2015}, PostgreSQL é um Sistema
Gerenciador de Banco de Dados Objeto-Relacional (SGBDOR) de código aberto
desenvolvido na universidade da Califórnia em Berkeley.

	\par Segundo \citeonline{milani2008}, a primeira versão foi lançada em 1987 com apoio
de órgãos como \textit{Army Research Office} (ARO) e a \textit{National Science
Foundation} (NSF). A primeira grande mudança ocorrida foi em 1994 criando o
Postgres95 com a vantagem da incorporação da linguagem SQL (\textit{Structured
Query Language}).

	\par \citeonline{biazus2003}, afirma que com a popularização rebatizaram–o de
PostgreSql com implementação e melhorias de recursos com padrão SQL.

	\par Conforme \citeonline{sourceforge2015}, o sistema suporta funcionalidades modernas
como:

\begin{itemize}
  \item Comandos complexos. 
  \item Chaves estrangeiras.
  \item Integridade transacional. 
\end{itemize}
	
	\par Os programadores podem melhora-lo criando funcionalidades como:
\begin{itemize}
  \item Funções.
  \item Operadores. 
  \item Tipos de dados. 
\end{itemize}

	\par O PostgreSQL é um \textit{software} de fácil utilização e multiplataformas
o que leva a ser implantado em muitas empresas.

















