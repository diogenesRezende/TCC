\chapter{QUADRO TEÓRICO}

	\par Neste capítulo serão descritos os principais conceitos e características
das tecnologias a serem utilizadas para o desenvolvimento dos \textit{softwares}
propostos nos objetivos dessa pesquisa.

\section{\textit{Java}}

	\par Segundo \citeonline{deitel2002} o java veio ao publico em 1995 pela
\textit{Sun Microsystem}. Os criadores dessa nova tecnologia liderados por
James Gosling basearam-se em duas linguagens muito utilizadas no mundo, C e C++.
Isso deu ao java uma base para implementar em novos sistemas como, sistemas
operacionais, sistemas de comunicações, sistema de banco de dados e aplicativos
para computadores pessoais.

	\par Entre as principais características pode – se dizer que o Java é:
	
	\begin{itemize}

	  \item Orientado a objeto.
	  
	  \item Seguro.
	  
	  \item Independente de plataforma. 

	\end{itemize}
	
	\par \citeonline{deitel2002} O \textit{java} será utilizado para codificar e
criar regras para proteger o banco de dados , gerenciando a infraestrutura que
o próprio java fornece como, Transação, Acesso remoto, Web services,
Gerenciamento de threads, Gerenciamento de conexões http\footnote{Http -
HyperText Transfer Protocol.}.

	\par Para acessar o servidor precisamos de um serviço \textit{web} (\textit{Web
services}) que utilizara o \textit{java} para criar as regras e serviços de
compilação de dados. Segundo \citeonline{deitel2002} existem 5 fases para que os dados
cheguem em seu dispositivo que são, armazenamento em disco, o compilador cria os
\textit{bytecodes}, transferência dos \textit{bytecodes} para memória, verifica
a integridade dos \textit{bytecodes} para que não haja nenhuma violação das
restrições de segurança e por ultimo o interpretador(JMV) lê os
\textit{bytecodes} e os traduz para a linguagem que o computador entenda e
possivelmente armazena os valores dos dados enquanto executa o programa.

	\par \citeonline{romanato2015} afirma, que o \textit{Java} usa a JMV
(\textit{Java Virtual Machine}), que é uma máquina virtual capaz de converter
os \textit{Byte Codes}\footnote{Byte Codes – Código em Bytes } para a linguagem
do sistema operacional utilizado pelo cliente, sem a necessidade de compila-lo
para cada plataforma. Dessa maneira um \textit{software} que é executado no
\textit{Windows}, funcionará normalmente em qualquer outro sistema.

\section{\textit{Android}}

	\par Segundo \citeonline{monteiro2012}, \textit{Android} é um sistema
operacional baseado em \textit{Linux}, de código aberto e que utiliza a
linguagem de programação \textit{Java} para o desenvolvimento de seus
aplicativos. Criado especialmente para dispositivos móveis, começou a
ser desenvolvido no ano de 2003 pela então empresa Android Inc, que em 2005 foi
agregada ao Google. A partir de 2007 o projeto \textit{Android} uniu-se a
\textit{Open Handset Alliance}, uma associação de empresas de
\textit{softwares}, \textit{hardwares} e telecomunicações, que tem por
finalidade desenvolver uma plataforma para dispositivos móveis que seja
completa, aberta e gratuita.

	\par \citeonline{krazit2009} publicou uma entrevista com Rubin, um dos
idealizadores do \textit{Android}, o qual afirma que o sistema pode rodar em
equipamentos de diversos fabricantes, evitando assim ficar limitado a poucos
dispositivos. Conforme informações do site \citeonline{android1}, hoje em dia
existe mais de um bilhão de aparelhos espalhados pelo mundo com esse sistema
operacional.

	\par De acordo com \citeonline{monteiro2012} as aplicações são executadas em
uma máquina virtual Java denominada \textit{Dalvik}. Cada aplicativo, usa uma
instância dessa máquina virtual tornando-a assim mais segura. Por outro lado os
\textit{softwares} só podem acessar algum recurso do dispositivo, como uma
lista de contatos, caso seja formalmente aceito pelo usuário nos termos de uso
ao instalá-lo.

	\par Dentre os muitos benefícios do Android, um recurso muito utilizado é a
chamada \textit{Intents}, que de acordo com \citeonline[p.29]{k192012}, "são
objetos responsáveis por passar informações, como se fossem mensagens, para os
principais componentes da API do Android, como as \textit{Activities},
\textit{Services} e \textit{BroadCast Receivers}". \citeonline{monteiro2012}
diz que as \textit{Intents} são criadas quando se tem a intenção de realizar
algo como por exemplo compartilhar uma imagem, utilizando os recursos já
existentes no dispositivo. Existem dois tipos de \textit{Intents}:
	
	\begin{itemize}
	  
	  \item \textit{Intents} implícitas - Quando não é informada qual
	  \textit{Activity} deve ser chamada, ficando assim por conta do sistema
	  operacional verificar qual a melhor opção.
	  
	  \item \textit{Intents} explicitas - Quando é informada qual
	  \textit{Activity} deve ser chamada. Usada normalmente para chamar
	  \textit{activities} da mesma aplicação.
	  
	\end{itemize}
	
	\par Segundo \citeonline{k192012} uma aplicação Android pode ser construída com
quatro tipos de componentes:

	\begin{itemize}
	  
	  \item \textit{Activity} - São as telas com interface gráfica, com capacidade
	  de interações com os usuários.
	  
	  \item \textit{Services} - São serviços executados em segundo plano, com
	  tarefas que levam algum tempo sem comprometer a interação do usuário. 
	  
	  \item \textit{Content Providers} - São os provedores de conteúdo que permitem
	  o acesso e a modificação de dados. 
	  
	  \item \textit{Broadcast Receivers} - São componentes capazes de receber e
	  responder a eventos do sistema operacional. 
	
	\end{itemize}
	  
	\par Em uma aplicação, um elemento fundamental é a interface gráfica, que
deverá ser organizada, simples e elegante. Conforme \citeonline{monteiro2012}
esses são os principais \textit{Layouts} do sistema operacional Android:
	\begin{itemize}
	
	 \item \textit{LinearLayout} - Permite posicionar os elementos em forma
	 linear, dessa forma quando o dispositivo estiver em forma vertical os itens
	 ficaram um abaixo do outro e quando estiver na posição horizontal eles
	 ficaram um ao lado do outro.
	
	  \item \textit{RelativeLayout} - Permite posicionar elementos de forma
	  relativa, ou seja um item com relação a outro.
	
	  \item \textit{TableLayout} - Permite criar \textit{layouts} em formato de
	  tabelas. O elemento \textit{TableRow} representa uma linha da tabela e seus
	  filhos as células.  Dessa maneira, caso um \textit{TableRow} possua dois
	  itens significa que essa linha tem duas colunas.
	
	  \item \textit{DatePicker} - \textit{Widget} desenvolvido para a seleção de
	  datas que podem ser usadas diretamente no \textit{layout} ou através de
	  caixas de diálogo.

	  \item \textit{Spinner} - \textit{Widget} que permite a seleção de itens. 
	
	  \item \textit{ListViews} - Permite exibir itens em uma listagem. Dessa
	  forma, em uma lista de compras, clicando em uma venda é possível listar os
	  itens dessa venda selecionada.
	
	  \item \textit{Menus} - Um item muito importante, pois apresenta aos usuários
	  as opções existentes no aplicativo.
	
	  \item \textit{AlertDialog} - Apresenta informações aos usuários através de
	  uma caixa de diálogo. Comumente utilizado para perguntar ao cliente o que
	  deseja fazer quando seleciona algum elemento.
	
	  \item \textit{ProgressDialog} e \textit{ProgressBar} - Utilizado quando uma
	  aplicação necessita de um recurso que será demorado, como por exemplo, fazer
	  um \textit{download}, pode ser feito uma animação informando ao usuário o
	  progresso da operação.
	
	\end{itemize}
	
	\par Para uma maior interação, as aplicações normalmente utilizam API’s de
terceiros, como o Google \textit{Maps}, quando necessita encontrar alguma
localização. Para \citeonline{monteiro2012} essa comunicação pode utilizar o
REST, que envia requisições através da URL. Ao receber informações pedidas a um
outro serviço, ela pode estar no padrão XML ou JSON. O REST será detalhado mais
adiante.

	\par Outra ferramenta importante e muito utilizada do Android é a Notificação.
Segundo \citeonline{phillips2013} quando uma aplicação está sendo executada em
segundo plano e necessita comunicar-se com o usuário, o aplicativo cria uma
notificação. Normalmente as notificações aparecem na barra superior, o qual
pode ser acessado arrastando para baixo a partir da parte superior da tela.
Assim que o usuário clica na notificação ela cria uma \textit{activity} abrindo
a aplicação em questão.

	\par Com a ideia de desenvolver um aplicativo para dispositivos móveis, a
plataforma Android foi escolhida devido ao seu destaque no mercado e pela
facilidade que apresenta aos usuários.

\section{Android Studio}

	\par Umas das ferramentas mais utilizadas para o desenvolvimento em Android é o
\textit{Eclipse IDE}, contudo a Google criou um \textit{software} especialmente
para esse ambiente, chamado \textit{Android Studio}. Segundo
\citeonline{gusmao2014}, \textit{Android Studio} é uma IDE baseado no ItelliJ
\textit{Idea} e foi apresentado na Conferência para desenvolvedores I/O de 2013.

	\par De acordo com \citeonline{hohensee2013} o \textit{Android Studio} tem um
sistema de construção baseado em \textit{Gradle}, que permite aplicar
diferentes configurações no código quando há necessidade de criar mais de uma
versão, como por exemplo, um \textit{software} que terá uma versão gratuita e
outra paga, melhorando a reutilização do código. Com o \textit{Gradle} também é
possível fazer os \textit{downloads} de todas as dependências de uma forma
automática sem a necessidade de importar bibliotecas.

	\par \citeonline{hohensee2013} afirma que o \textit{Android Studio} é um editor
de código poderoso, pois tem como característica a edição inteligente, pois ao
digitar já completa as palavras reservadas do sistema operacional e fornece uma
organização do código mais legível.

	\par Segundo \citeonline{android2} a IDE tem suporte para a edição de
interface, o que possibilita ao desenvolvedor arrastar os componentes que
deseja. Ao testar o aplicativo permite o monitoramento do consumo de memória e
de processador por parte do utilitário.

	\par \citeonline{gusmao2014} diz que a plataforma tem uma ótima integração com
o \textit{GitHub} e está disponível para \textit{Windows}, \textit{Mac} e
\textit{Linux}. Além disso os programadores terão disponíveis uma versão
estável e mais três versões que estarão em teste chamadas de \textit{Beta},
\textit{Dev} e \textit{Canary}.

	\par Devido ao \textit{Android Studio} ser uma ferramenta de fácil usabilidade
e a IDE oficial para o desenvolvimento Android, esta foi escolhida como ambiente de
construção do aplicativo.	
	
\section{\textit{Web Services}}
	
	\par Nos tempos atuais, com o grande fluxo de informação que percorre pelas
redes da \textit{internet} é necessário um nível muito alto de integração entre
as diversas plataformas, tecnologias e sistemas. Como uma provável solução para
esse ponto, já existem as tecnologias de sistemas distribuídos. Porém essas
tecnologias sofrem demasiadamente com o alto acoplamento de seus componentes e
também com a grande dependência de uma plataforma para que possam funcionar. Com
intuito de solucionar a estes problemas e proporcionar alta transparência entre
as várias plataformas, foram criados as tecnologias \textit{web services}.
	
	\par De acordo com \citeonline{erl2015}:
	\begin{citacao}
		No ano de 2000, a W3C (\textit{World Wide Web Consortium}) aceitou a submissão
		do \textit{Simple Object Access Protocol} (SOAP). Este formato de mensagem
		baseado em XML estabeleceu uma estrutura de transmissão para comunicação entre
		aplicações (ou entre serviços) via HTTP. Sendo uma tecnologia não amarrada a
		fornecedor, o SOAP disponibilizou uma alternativa atrativa em relação aos
		protocolos proprietários tradicionais, tais como CORBA e DCOM.
	\end{citacao}
	
	\par Considera-se então a existência dos \textit{web services} a partir daí. De
acordo com \citeonline{duraes2005}, \textit{Web Service} é um componente que
tem por finalidade integrar serviços distintos. O que faz com que ele se torne
melhor que seus concorrentes é a padronização do XML(\textit{Extensible Markup
Language}) para as trocas de informações. A aplicação consegue conversar com o
servidor através do WSDL que é documento que contém as regras de funcionamento
do \textit{web service}.
	
	\par Os \textit{web services} além de fornecerem uma padronização de
comunicação entre as várias tecnologias existentes, provê transparência na
troca de informações. Isso contribui pelo fato de que as novas aplicações
possam se comunicar com aplicações mais antigas ou aplicações contruídas sobre
outras plataformas.

	\par Além das tecnologias \textit{web services} tradicionais, existe também os
\textit{web services} REST que também disponibilizam serviços, porém não
necessitam de encapsulamento de suas mensagens assim como os \textit{web
Services} SOAP. Este fato influencia diretamente na performance da aplicação
como um todo, haja vista que não sendo necessário o encapsulamento da informação
requisitada ao \textit{web service}, somente é necessário o processamento e
tráfego da informação que realmente importa. As caracteristícas do padrão REST
serão abordadas a seguir.

	\subsection{REST}
	
	\par Segundo \citeonline{saudate2012}, REST é a sigla de
\textit{Representational State Transfer}, ou em português Transferência de
Estado Representativo, desenvolvido por Roy Fielding na defesa de sua tese de
doutorado. Segundo o próprio \citeonline{fielding2000} REST é um estilo que
deriva do vários estilos arquitectónicos baseados em rede e  que combinado
com algumas restrições, fornecem uma interface simples e uniforme para
fornecimento de serviços\footnote{Tradução e resumo de informações de
responsabilidade do autor da pesquisa.}.
	
	\par \citeonline{rubbo2015}, afirma que os dados e funcionalidade de um sistema
são considerados recursos e podem ser acessados através das URI's
\textit{(Universal Resource Identifier)}, facilitando dessa forma a comunicação
do servidor com o cliente.
	 	
	\par \citeonline{saudate2012}, explica ainda que os métodos do HTTP\footnote{
 HTTP – Abreviação para \textit{HyperText Transfer Protocol}} podem fazer
 modificações nos recursos através dos comandos:
	 
	 \begin{itemize}
	   \item GET – Para recuperar algum dado. 
	   \item POST – Para criar algum dado.
	   \item PUT – Para alterar algum dado. 
	   \item DELETE – Para excluir algum dado. 
	 \end{itemize}
	 
	\par Segundo \citeonline{godinho2009}, não há um padrão de formato para as
 trocas de informações, mas as que mais são utilizadas é o XML\footnote{XML –
 Abreviação para \textit{Extensible Markup Language}.} e o JSON\footnote{JSON –
 Abreviação para \textit{JavaScript Object Notation}.}. O REST\footnote{REST –
 Abreviação para \textit{Representational State Transfer}.} é o mais indicado
 para aplicações em dispositivos moveis, devido sua agilidade.
	
	
\section{\textit{Apache Tomcat}}

	\par De acordo com a \citeonline{tomcat2015}, \textit{Apache Tomcat} é uma
implementação de código aberto das especificações \textit{Java Servlet} e
\textit{JavaServer Pages}. O \textit{Apache Tomcat}  é um \textit{Servlet
Container}, que disponibiliza serviços através de requisições e respostas.
	
	\par Segundo \citeonline{tomcat2015}, o projeto desse \textit{software} começou
com a \textit{Sun Microsystems}, que em 1999 doou a base do código para
\textit{Apache Software Foundation} onde foi lançada a versão 3.0.
	
	\par Desta forma, o cliente envia uma requisição através do seu navegador, o
servidor por sua vez a recebe, executa o \textit{servlet} e devolve a resposta
ao usuário.

\section{PostgreSQL}

\par Segundo \citeonline{milani2008} O postgreSQL é um SGBD(Sistema de
Gerenciamento de Banco de Dados) que tem suporte para ACID(Atomicidade,
consistência, isolamento e Durabilidade), que são serviços que garantem a
qualidade que um banco de dados. A seguir algumas das principais
características e recursos existentes no postgreSQL, que são, Replicação,
Cluster(Alta disponibilidade), Multithreads, segurança ssl e criptografia, sql.

\par Segundo \citeonline{milani2008}:
	\begin{itemize}
	  
	  \item Replicação: É o compartilhamento de processos e distribuição das
	  informações em diferentes bancos de dados. Ou seja as informações que serão
	  armazenadas no servidor serão replicadas para um servidor secundário,
	  mantendo os dados íntegros.
	  
	  \item Cluster: É a interligação de dois ou mais computadores e a
	  sincronização entre eles, assim aumentando a capacidade de demanda do banco
	  de dados. 
	  
	  \item Multithreads: É a manipulação de dados de forma que mais de uma pessoa
	  tenha acesso a mesma informação sem ocasionar atrasos ou filas de acessos.
	  
	  \item Segurança SSl e criptografia: Possibilita criar conexões seguras,
	  tanta para trafegar informações de login quando aquelas consideradas sigilosas.
   
	\end{itemize}
	
	\par \citeonline{stones2005beginning}  diz que um dos pontos fortes do
postgreSQL deriva-se de sua arquitetura onde pode ser usado em um ambiente
cliente/servidor, beneficiando o usuário e o desenvolvedor.

	\par Segundo \citeonline{stones2005beginning} postgreSQL é comparado com
qualquer outro sgbd, contendo todas as características que encontraria em outro
banco de dados, e algumas características que não encontra em outro
\textit{software} como, transações, subconsultas, chave estrangeira e regras de
herança.

	\par O PostgreSQL é um \textit{software} de fácil utilização e multiplataforma
o que leva a ser implantado em muitas empresas.


\section{Engenharia De \textit{Software}}

	\par 
	
	\subsection{UML}
		
		\par 
	
	\subsection{Processos de Desenvolvimento de \textit{Software}}
	