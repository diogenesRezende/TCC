\section{\textbf{Google Cloud 	Messaging}}

	\par Para que os graduandos sejam notificados quando houver alguma mudança no
portal do aluno será utilizada uma API oferecida pela \textit{Google}
denominada \textit{Google Cloud Messaging} ou simplesmente
GCM\footnote{\textit{Google Cloud Messaging}}, um recurso que tem por objetivo
notificar as aplicações Android. Segundo \citeonline{leal2014}, ele permite que
aplicações servidoras possam enviar pequenas mensagens de até 4
KB\footnote{\textit{KB - Kilobytes}} para os aplicativos móveis, sem que este
necessite estar em execução. Ainda de acordo com \citeonline{leal2014} para o
bom funcionamento do recurso apresentado, são necessários os seguintes componentes:

\begin{itemize}
	
	\item \textit{Sender ID\footnote{Identity}}: é o identificador do projeto.
	Será utilizado pelo servidores da \textit{Google} para identificar a aplicação
	que envia a mensagem.
	
	\item \textit{Application ID}: é o identificador da aplicação Android. O
	identificador é o nome do pacote do projeto que consta no AndroidManifest.xml.
	
	\item \textit{Registration ID}: é o identificador gerado pelo servidor GCM
	quando aplicação Android se conecta a ele. Este deve ser enviado também a
	aplicação servidora.
	
	\item \textit{Sender Auth Token}: é uma chave que é incluída no cabeçalho
	quando a mensagem é enviada da aplicação servidora para o GCM. Essa chave é
	para que a API da Google possa enviar as mensagens para o aplicativo correto.

\end{itemize}

	\par De acordo com os componentes acima citados, quando uma aplicação servidora
enviar uma mensagem para o aplicativo Android, na verdade está enviando para o
servidor GCM que será encarregado de enviar a mensagem para a aplicação mobile.