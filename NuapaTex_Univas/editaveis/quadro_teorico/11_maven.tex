%\section{Maven}
	
	\par Maven é uma famosa ferramenta de automação de compilação e gerência de
dependências para projetos Java. Nasceu com o objetivo de simplificar o processo
de compilação do projeto Jakarta Turbine. Possui inúmeras funcionalidades, sendo
que as mais comuns são gerência de dependências de um projeto e os processos de
\textit{build} do mesmo.
	
	\par \citeonline{apacheMaven2015} afirma que Maven é baseado no conceito de um
\textit{Project Object Model} (POM), e pode gerenciar um projeto de
construção, elaboração de relatórios e documentação de um software.
\citeonline{lecheta2015} ainda complementa ao afirmar que , "a configuração de
um projeto Maven é feita no arquivo pom.xml, que descreve os dados de seu
projeto, suas de dependências e varias outras configurações".
	
	\par A principal funcionalidade do Maven é a gerência de dependências de um
projeto Java. \citeonline{lecheta2015} afirma que "ao adicionar uma dependência
ao projeto, o Maven faz o \textit{download} desta dependência diretamente de
um repositório mundial de projetos  conhecido com Maven Central". Se necessário
ainda resolve dependências das dependências do projeto. De acordo com
\citeonline{apacheMaven2015} existem três ciclos de vida de compilação para um
projeto que usa Maven:

	\begin{itemize}
		  \item Ciclo de vida \textit{default}: lida com a implantação do projeto;
		  \item Ciclo de vida \textit{clean}: lida com a limpeza do projeto;
		  \item Ciclo de vida \textit{site}: lida com a criação de documentação do
		  local do seu projeto. 
	\end{itemize}
	
	\par Com o intuito de facilitar o desenvolvimento do \textit{web service},
haja vista a quantidade muito elevada de dependências, a ferramenta Maven foi
usada.
