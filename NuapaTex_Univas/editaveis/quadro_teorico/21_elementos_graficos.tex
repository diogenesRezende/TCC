%\subsection{Elementos Gráficos}
	
	\par Em uma aplicação, um elemento fundamental é a interface gráfica, que
deverá ser organizada, simples e elegante. O Android organiza os elementos
gráficos (\textit{widgets}) através de \textit{layouts}. Conforme
\citeonline{monteiro2012} esses são os principais \textit{layouts} do sistema
operacional Android:

	\begin{itemize}
	
		 \item \textit{LinearLayout}: permite posicionar os elementos em forma
		 linear. Dessa forma se o \textit{layout} for configurado com orientação
		 vertical os itens ficaram um abaixo do outro, porém se estiver configurado
		 com orientação horizontal eles ficaram um ao lado do outro.
		
		  \item \textit{RelativeLayout}: permite posicionar elementos de forma
		  relativa, ou seja um item com relação a outro.
		
		  \item \textit{TableLayout}: permite criar \textit{layouts} em formato de
		  tabelas. O elemento \textit{TableRow} representa uma linha da tabela e seus
		  filhos as células. Dessa maneira, caso um \textit{TableRow} possua dois
		  itens significa que essa linha tem duas colunas.
		
		  \item \textit{DatePicker}: \textit{widget} desenvolvido para a seleção de
		  datas que podem ser usadas diretamente no \textit{layout} ou através de
		  caixas de diálogo.
	
		  \item \textit{Spinner}: \textit{widget} que permite a seleção de itens,
		  similar ao \textit{combobox}.
		
		  \item \textit{ListView}: permite exibir itens em uma listagem. Dessa
		  forma, em uma lista de compras, clicando em uma venda, é possível ver os
		  detalhes do item selecionado.
		
		  \item \textit{Action Bar}: um item muito importante, pois apresenta na parte
		  superior aos usuários as opções existentes no aplicativo.
		
		  \item \textit{AlertDialog}: apresenta informações aos usuários através de
		  uma caixa de diálogo. Comumente utilizado para perguntar ao cliente o que
		  deseja fazer quando ele seleciona algum elemento.
		
		  \item \textit{ProgressDialog} e \textit{ProgressBar}: utilizado quando uma
		  aplicação necessita de um recurso que levará um certo tempo para executar,
		  como por exemplo, fazer um \textit{download}, pode ser feito uma animação
		  informando ao usuário o progresso da operação.
		  
		  \item \textit{ExpandableListView}: exibe os itens em forma de uma lista
		  como o \textit{listView}, o que diferencia-o é que ele mostra uma lista de
		  dois níveis de rolagem vertical, em vez de abrir uma outra tela.

	\end{itemize}

\par Com a ideia de desenvolver um aplicativo para dispositivos móveis, a
plataforma Android foi escolhida devido ao seu destaque no mercado e pela
facilidade que apresenta aos usuários e desenvolvedores.