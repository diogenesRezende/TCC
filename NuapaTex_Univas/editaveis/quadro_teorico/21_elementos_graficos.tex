%\subsection{Elementos Gráficos}
	
	\par Em uma aplicação, um elemento fundamental é a interface gráfica, que
deverá ser organizada, simples e elegante. O Android organiza os elementos
gráficos (\textit{widgets}) através de \textit{layouts} Conforme
\citeonline{monteiro2012} esses são os principais \textit{layouts} do sistema
operacional Android:
	
		\begin{itemize}
		
			\item LinearLayout: permite posicionar os elementos em forma linear, dessa
			forma quando o dispositivo estiver em forma vertical os itens ficarão um
			abaixo do outro e quando estiver na horizontal eles ficarão um ao lado do
			outro.
			
			\item RelativeLayout: permite posicionar elementos de forma relativa, ou
			seja um \textit{widget} com relação a outro.
			
			\item TableLayout: permite criar \textit{layouts} em formato de tabelas. O
			elemento TableRow representa uma linha da tabela e seus filhos são as
			células.  Dessa maneira, caso um TableRow possua dois itens, significa que
			essa linha tem duas colunas.
			
			\item DatePicker: \textit{widget} desenvolvido para a seleção de datas que
			podem ser usadas diretamente no \textit{layout} ou através de caixas de
			diálogo.
		
			\item Spinner: \textit{widget} que permite a seleção de itens, similar ao
			\textit{combobox}.
			
			\item ListViews: permite exibir itens em uma listagem. Dessa forma, em uma
			lista de compras, clicando em uma venda, é possível listar os detalhes dessa
			venda selecionada.
			
			\item Action Bar: um item muito importante, pois apresenta na parte superior
			aos usuários as opções existentes no aplicativo.
			
			\item AlertDialog: apresenta informações aos usuários através de uma caixa
			de diálogo. Comumente utilizado para perguntar ao cliente o que deseja fazer
			quando ele seleciona algum elemento.
			
			\item ProgressDialog e ProgressBar: utilizado quando uma aplicação necessita
			de um recurso que levará um certo tempo para executar, como por exemplo,
			fazer um \textit{download}, pode ser feito uma animação informando ao
			usuário o progresso da operação.
			 
			\item SQLite: é um banco de dados embarcado na plataforma Android, que
			armazena tabelas, \textit{views}, índices, \textit{triggers} em apenas um
			arquivo. Somente é possível acessá-lo pela aplicação a qual o criou, e é
			exluído caso o aplicativo seja removido.
	
		\end{itemize}
	
	\par Além dos recursos acima citados, um outro \textit{widget} que pode-se
destacar é o ExpandableListView, que para \citeonline{android3}, exibe os itens
em forma de uma lista similar ao ListView, o que diferencia-o é que ele mostra
uma lista de dois níveis de rolagem vertical, em vez de abrir uma outra tela.

	\par Outra ferramenta importante e muito utilizada do Android é a notificação.
Segundo \citeonline{phillips2013}, quando uma aplicação está sendo executada em
segundo plano e necessita comunicar-se com o usuário, o aplicativo cria uma
notificação. Normalmente as notificações aparecem na barra superior, o qual
pode ser acessado arrastando para baixo. Assim que o usuário clica na
notificação, ela cria uma Intent para abrir a aplicação em questão.

	\par Com a ideia de desenvolver um aplicativo para dispositivos móveis, a
plataforma Android foi escolhida devido ao seu destaque no mercado e pela
facilidade que apresenta aos usuários e desenvolvedores.