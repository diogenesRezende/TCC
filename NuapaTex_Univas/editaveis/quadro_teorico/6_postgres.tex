\section{PostgreSQL}

	\par Para \citeonline{milani2008}, todas as aplicações que armazenam
informações para o seu uso posterior devem estar integradas a um banco de
dados, seja armazenando em arquivos de textos ou em tabelas. Por isso, o
\textit{PostgreSql} tem por finalidade armazenar e administrar os dados em uma
solução de informática.
	
	\par \citeonline[s.p]{postgresWiki2015} define que “o \textit{PostgreSql} é
um SGBD (Sistema Gerenciador de Banco de Dados) objeto-relacional de código
aberto, com mais de 15 anos de desenvolvimento. É extremamente robusto e
confiável, além de ser extremamente flexível e rico em recursos.” 

	\par Conforme afirma \citeonline{milani2008}, o PostgreSql é um
SGDB\footnote{SGDB - Sistema Gerenciador de Banco de Dados } de código aberto
originado na Universidade de \textit{Berkeley}, na Califórnia (EUA) no ano de
1986, pelo projeto \textit{Postgres} desenvolvido por uma equipe sob liderança
do professor Michael Stonebraker. Ele possui os principais recursos dos bancos
de dados pagos e está disponível para os sistemas operacionais
\textit{Windows}, \textit{Linux} e \textit{Mac}. Atualmente existem bibliotecas
e \textit{drivers} para um grande número de linguagens de programação, entre as
quais podemos citar: C/C++, PHP, \textit{Java}, ASP, \textit{Python} etc.

	\par De acordo com \citeonline{postgres2015}, existem sistemas com o
\textit{PostgreSql} que gerenciam até quatro \textit{terabytes} de dados. Seu
banco não possui um tamanho máximo e nem um número máximo de linhas por tabela.
Contudo, uma tabela pode chegar a ter um tamanho de trinta e dois
\textit{terabytes} e cada campo a um \textit{gigabyte} de informação.

	\par Segundo \citeonline{milani2008}, são características do
\textit{PostgreSql}:

	\begin{itemize}
	  
	  \item Suporte a ACID (Atomicidade, Consistência, Isolamento e Durabilidade). 
	  
	  \item Replicação de dados entre servidores.
	  
	  \item Cluster.
	  
	  \item Multithreads.
	  
	  \item Segurança SSL\footnote{SSL -\textit{Secure Socket Layer}} e
	  criptografia.
	    
	\end{itemize}

	\par É através do \textit{PostgreSql} que o \textit{webservice} armazenará e
posteriormente retornará os dados dos discentes para o aplicativo
\textit{Andorid}.