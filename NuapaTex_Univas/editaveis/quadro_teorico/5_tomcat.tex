\section{\textit{Apache Tomcat}}

	\par De acordo com \citeonline{tomcat2015}, \textit{Apache Tomcat} é uma
implementação de código aberto das especificações \textit{Java Servlet} e
\textit{JavaServer Pages}. O \textit{Apache Tomcat} é um \textit{Servlet
Container}, que disponibiliza serviços através de requisições e respostas.
\citeonline{caelum2} afirma que ele utilizado para aplicações que necessitam
apenas da parte \textit{Web} do Java EE\footnote{EE - Sigla para enterprise
edition}.

	\par Segundo \citeonline{tomcat2015}, o projeto desse \textit{software}
começou com a \textit{Sun Microsystems}, que em 1999 doou a base do código para
\textit{Apache Software Foundation}, e então seria lançada a versão 3.0.

	\par Conforme \citeonline{devMedia2015}, para o desenvolvimento com
	\textit{Tomcat} é necessária a utilização das seguintes técnologias:
	
	\begin{itemize}
	  
	  \item JAVA: é utilizado em toda parte lógica da aplicação.
	  
	  \item HTML: é utilizado na parte de interação com o usuário.
	  
	  \item XML: é utilizado para as configurações do \textit{software}. 
	
	\end{itemize}
 
 
 
	\par Desta forma, o cliente envia uma requisição através do seu navegador, o
servidor por sua vez a recebe, executa o \textit{servlet} e devolve a resposta
ao usuário.