\section{Engenharia de \textit{Software}}

	\par De acordo com \citeonline{carvalho2001}, a engenharia de \textit{software}
surgiu na década de 80, com intuito de melhorar o desenvolvimento de
\textit{software}, produzindo sistemas de alta qualidade com a redução do custo
e do tempo.

	\par Segundo \citeonline[p.39]{pressman2011}, engenharia de \textit{software} é
“o estabelecimento e o emprego de sólidos princípios da engenharia de modo a
obter \textit{software} de maneira econômica, que seja confiável e funcione de
forma eficiente em máquinas reais”.

	\par Como afirmam \citeonline{carvalho2001}, a engenharia possui modelos de
processos que possibilitam ao gerente controlar o desenvolvimento e aos
programadores uma base para produzir. Alguns desses paradigmas são:

	\begin{itemize}
	  
	  \item Ciclo de vida clássico: utiliza o método sequencial, em que o final
	  de uma fase é o início da outra.
	  
	  \item O paradigma evolutivo: baseia-se no desenvolvimento e implementação
	  de um produto inicial. Esse produto passa por críticas dos usuários e vai
	  recebendo melhorias e versões até chegar ao produto desejado.
	  
	  \item O paradigma espiral: engloba as melhores características do ciclo de
	  vida clássica e o paradigma evolutivo. Ele consiste em vá - rios ciclos e
	  cada ciclo representa uma fase do pesquisa.
	
	\end{itemize}



	\par De toda a engenharia de \textit{software}, o que mais será utilizado nesse
projeto é a linguagem UML, que através dos seus diagramas norteará os caminhos
a serem seguidos.