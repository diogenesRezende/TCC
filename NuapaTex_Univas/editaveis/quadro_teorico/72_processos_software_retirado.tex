\subsection{Processos de \textit{Software}}
	
	
	\par Segundo \citeonline[p.52]{pressman2011}, um processo de \textit{software} é
“uma metodologia para as atividades, ações e tarefas necessárias para
desenvolver um \textit{software} de alta qualidade”.

	\par Para \citeonline{sommerville2003}, não existe um processo ideal, pois isso
dependerá de cada projeto, possibilitando cada qual implementar algum modelo já
existente. Contudo \citeonline{pressman2011} afirma que uma metodologia
genérica possui cinco passos:
	\begin{itemize}
	  
	  \item Comunicação: antes de iniciar os trabalhos técnicos deve-se entender
	  os objetivos do sistema e levantar requisitos para o bom funcionamento do
	  \textit{software}.
	  
	  \item Planejamento: cria um plano de projeto, que conterá as tarefas a
	  serem seguidas, riscos prováveis e recursos necessários.
	  
	  \item Modelagem: esboça o sistema para que se tenha uma ideia de como ele
	  deverá ficar e como encontrar a melhor solução para desenvolvê-lo.
	  
	  \item Construção: é a etapa de desenvolvimento e testes.
	  
	  \item Emprego: o \textit{software} pronto em sua totalidade ou parcialmente
	  é implantado no cliente e este retorna o seu \textit{feedback}.
	 
	 \end{itemize}
	 
	 \par Dessa forma, normalmente qualquer um dos modelos (ciclo de vida clássico,
evolutivo ou espiral) utilizaram os princípios das metodologias acima citadas.