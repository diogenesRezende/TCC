	\subsection{REST}
	
			\par Segundo \citeonline{saudate2012}, REST\footnote{REST -
			\textit{Representational State Transfer} ou Transferência de Estado
			Representativo.}, desenvolvido por Roy Fielding na defesa de sua tese de
			doutorado. Segundo o próprio \citeonline{fielding2000} REST é um estilo que
			deriva dos vários estilos arquitetônicos baseados em rede e  que combinado
			com algumas restrições, fornecem uma interface simples e uniforme para
			fornecimento de serviços\footnote{Tradução e resumo de informações de
			responsabilidade dos autores da pesquisa.}.
			
			\par \citeonline{rubbo2015} afirma que os dados e funcionalidades de um
			sistema são considerados recursos e podem ser acessados através das URI's
		\textit{(Universal Resource Identifier)}, facilitando dessa forma a comunicação
		do servidor com o cliente. Um serviço contruído na arquitetura REST
		basea-se fortemente em recursos. Para exemplificar o que seria um recurso em
		REST temos o seguinte cenário: considera-se uma URL tal como 
		http://www.univas.edu.br/alunos/ onde pretende-se fazer a gerência dos dados
		de um aluno ou de um conjunto de alunos. A recuperação de dados, bem como sua
		edição e/ou deleção fazendo uso pleno dos métodos HTTP, através dessa URL,
		pode ser considerado como um recurso.
			\par \citeonline{saudate2012}, explica ainda que os métodos do HTTP podem fazer
		modificações nos recursos, da seguinte forma:
			
			 \begin{itemize}
			   \item GET: para recuperar algum dado. 
			   \item POST: para criar algum dado.
			   \item PUT: para alterar algum dado. 
			   \item DELETE: para excluir algum dado. 
			 \end{itemize}
			 	
			\par Como o próprio \citeonline{fielding2000} também foi um dos criadores de um
		dos protocolos mais usados na web, o HTTP, pode-se dizer que o REST foi
		concebido para rodar sobre esse protocolo com a adição de mais algumas
		características que segundo \citeonline{saudate2013} foram responsáveis pelo
		sucesso da web. Essas características são:
		
			\begin{itemize}
			  
			  \item URLs bem definidas para recursos;
			  
			  \item Utilização dos métodos HTTP de acordo com seus propósitos;
			  
			  \item Utilização de \textit{media types} efetiva;
			  
			  \item Utilização de \textit{headers} HTTP de maneira efetiva;
			  
			  \item Utilização de códigos de status HTTP;
			  
			  \item Utilização de hipermídia como motor de estado da aplicação.
			
			\end{itemize}
			 
			\par Segundo \citeonline{godinho2009}, não há um padrão de formato para as
		 trocas de informações, mas as que mais são utilizadas é o XML\footnote{XML
		 - \textit{Extensible Markup Language}.} e o JSON\footnote{JSON - 
		 \textit{JavaScript Object Notation}.}. O REST é o mais indicado para aplicações
		 em dispositivos moveis, devido a sua agilidade.