\section{\textit{Jersey}}

	\par Atualmente um padrão para desenvolvimento de serviços \textit{web} vem
sendo bastante adotado, trata-se do padrão arquitetural REST. De acordo com
\citeonline{saudate2012} linguagem \textit{Java} possui uma especificação
própria para desenvolvimento de serviços REST desde de setembro de 2008, que é
a JSR311, ou como é popularmente chamado JAX-RS. Esta especificação provê um
conjunto de API's simples, para facilitar o desenvolvimento de serviços
\textit{web}. De acordo com \citeonline{oracle22015} "JAX-RS é uma API da
linguagem de programação \textit{Java} projetado para tornar mais fácil para
desenvolver aplicações que usam a arquitetura REST"\footnote{Tradução e resumo
de informações de responsabilidade dos autores da pesquisa.}. Através desta
especificação torna-se mais facíl e ágil a contrução de serviços \textit{web}
baseados em REST.

	\par Como JAX-RS é apena uma especificação, ela necessita então de uma
implementação. Uma das implementações desta especificação é o
\textit{framework Jersey}. Segundo \citeonline{oracle2015} "\textit{Jersey}, a
implementação de referência de JAX-RS, implementa suporte para as anotações
definidas no JSR 311, tornando mais fácil para os desenvolvedores a construir
serviços \textit{Web RESTful} usando a linguagem de programação
\textit{Java}"\footnote{Tradução e resumo de informações de responsabilidade
dos autores da pesquisa.}. Além das anotações que facilitam seu uso,
\textit{Jersey} pode prover serviços com uma infinidade muito grande de tipos de
mídias, tais como XML e JSON entre outros.

	\par O \textit{framework Jersey} tem amplo suporte para os varios métodos HTTP.
Fazendo uso dele pode-se facilmente implementar recursos REST. Além disso
\textit{Jersey}  pode rodar tanto em servidores que implementem a especificação
\textit{Servlet} ou não. Este \textit{framework} será usado para contruir o que
seria a parte responsável por prover os serviços para o aplicativo
\textit{Android}.