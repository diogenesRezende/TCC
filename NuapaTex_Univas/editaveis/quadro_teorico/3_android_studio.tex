\section{Android Studio}

	\par Umas das ferramentas mais utilizadas para o desenvolvimento em Android é o
\textit{Eclipse IDE}, contudo a Google criou um \textit{software} especialmente
para esse ambiente, chamado \textit{Android Studio}. Segundo
\citeonline{gusmao2014}, \textit{Android Studio} é uma IDE baseado no ItelliJ
\textit{Idea} e foi apresentado na conferência para desenvolvedores I/O de 2013.

	\par De acordo com \citeonline{hohensee2013}, o \textit{Android Studio} tem um
sistema de construção baseado em \textit{Gradle}, que permite aplicar
diferentes configurações no código quando há necessidade de criar mais de uma
versão, como por exemplo, um \textit{software} que terá uma versão gratuita e
outra paga, melhorando a reutilização do código. Com o \textit{Gradle} também é
possível fazer os \textit{downloads} de todas as dependências de uma forma
automática sem a necessidade de importar bibliotecas manualmente.

	\par \citeonline{hohensee2013} afirma que o \textit{Android Studio} é um editor
de código poderoso, pois tem como característica a edição inteligente, que ao
digitar já completa as palavras reservadas do \textit{Android} e fornece uma
organização do código mais legível.

	\par Segundo \citeonline{android2}, a IDE tem suporte para a edição de
interface, o que possibilita ao desenvolvedor arrastar os componentes que
deseja. Ao testar o aplicativo, ela permite o monitoramento do consumo de
memória e de processador por parte do utilitário.

	\par \citeonline{gusmao2014} diz que a plataforma tem uma ótima integração com
o \textit{GitHub} e está disponível para \textit{Windows}, \textit{Mac} e
\textit{Linux}. Além disso os programadores terão disponíveis uma versão
estável e mais três versões que serão em teste, chamadas de \textit{Beta},
\textit{Dev} e \textit{Canary}.

	\par Devido a fácil usabilidade e por ser a IDE oficial para o desenvolvimento
\textit{Android}, escolheu-se esse ambiente para a construção do aplicativo.