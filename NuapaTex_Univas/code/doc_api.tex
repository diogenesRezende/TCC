\begin{enumerate}
	\item \textbf{Contexto da Aplicação} -
	http://<enderecoDoServidor>/WebServiceAppUnivas/
		\begin{itemize}
			\item \textbf{Contextos internos}
				\begin{itemize}
					\item \textbf{students} - contexto relacionados aos dados dos alunos
							\begin{itemize}
								
								\item \textbf{events} - recurso referente aos dados sobre eventos
								relacionados a alunos;
									\begin{itemize}
									  \item \textbf{GET} - recebe como parâmetro a matrícula do aluno
									  através da url, e devolve um JSON, com todos os eventos do semestre corrente
									  relacionados ao aluno;\\ Ex.: 
									  http://<enderecoDoServidor>/WebServiceAppUnivas/students/\\events/98004095
									  \item \textbf{POST} - Não Implementado
									  \item \textbf{PUT} - Não Implementado
									  \item \textbf{DELETE} - Não Implementado
									\end{itemize}									
								\item \textbf{disciplines} - recurso referente aos dados sobre
								disciplinas de alunos;
									\begin{itemize}
									  \item \textbf{GET} - recebe como parâmetro a matrícula do aluno
									  através da url, e devolve um JSON, com todas as disciplinas do semestre
									  corrente relacionadas ao aluno;\\ Ex.: 
									  http://<enderecoDoServidor>/WebServiceAppUnivas/students/\\disciplines/98004095
									  \item \textbf{POST} - Não Implementado
									  \item \textbf{PUT} - Não Implementado
									  \item \textbf{DELETE} - Não Implementado
									\end{itemize}
							\end{itemize}
					\item \textbf{users} - contexto relacionado aos dados dos usuários para ser
					utilizado no futuro para login
						\begin{itemize}
									  \item \textbf{GET} - Não Implementado
									  \item \textbf{POST} - Não Implementado
									  \item \textbf{PUT} - recebe um JSON com os dados dos usuário e
									  retorna um cabeçalho com o código de  \textit{status} 200;
									  \item \textbf{DELETE} - Não Implementado
									\end{itemize}
				\end{itemize}
		\end{itemize}
\end{enumerate}
